Professional maturity in academia would be ideally reached by leading a dynamic research group actively working on fundamental problems at the interface of game theory and online learning, with strong impact in real-world applications. Dr Gabillon has shown an extremely high potential to achieve this goal. As evident from his solid publication record, he has strong expertise in the domain, always giving equal importance to theory and applications. The fellow has also demonstrated strong ability to acquire new knowledge  and become highly productive in a short period of time. Indeed, a significant result of his PhD thesis is in bringing classical reinforcement learning algorithms closer to daily life. Moreover, aside from providing theoretical guarantees for his proposed methods, this entailed spending a significant amount of time single-handedly managing extensive parallel-computing experiments over a grid of computers, a task for which he had no prior knowledge. During the course of his PhD, through a 6-months internship at a major US R\&D lab (Technicolor Research Laboratory, Palo Alto), he had the opportunity to collaborate with a new team of R\&D researchers. He quickly became productive and his efforts in this short period of time have resulted in the publication of two peer-reviewed papers at prestigious international conferences in machine learning. Through this experience he has also obtained valuable knowledge about industrial research and its interaction with academia. This has given him the ability to better understand the research pathways to produce high-impact results and establish significant collaborations with industry.

At the start of the fellowship, Dr Gabillon will be closely mentored by Professor Leslie at Lancaster University. He will also have access to the university's research resources, and will be able to further develop his research and supervision skills, which will greatly contribute to achieving professional maturity. At Lancaster University, the fellow will also have the unique opportunity to establish inter-disciplinary collaborations through the recently established STOR-i program, a quality research training interface between statistics and industry. 