\documentclass[a4paper,11pt]{article}

\usepackage[T1]{fontenc}
\usepackage{lmodern}


\usepackage[utf8]{inputenc}
\usepackage[french,english]{babel}

\usepackage{eurosym}
\usepackage{lastpage}
\usepackage{xspace}
\usepackage[margin=16mm,includehead,includefoot]{geometry}
\usepackage{fancyhdr}
\usepackage{booktabs,lipsum}
\usepackage{graphicx}
\usepackage{multirow}
\usepackage{array}
\usepackage{tabularx}
\usepackage{xcolor}
\usepackage{csquotes}
\usepackage{pgfgantt}
\usepackage{titlesec}
\usepackage[style=verbose-ibid,backend=bibtex]{biblatex}
\usepackage{hyperref}
\usepackage{amsmath}
\usepackage{amsfonts}
\usepackage{amssymb}
\usepackage{vgResume}

\newcommand{\TODO}[1]{{\textcolor{red}{[\textbf{TODO:} #1]}}}
\newcommand{\acronym}{{\sc OSEGA}\xspace}
\titlespacing\section{0pt}{6pt plus 4pt minus 2pt}{0pt plus 2pt minus 2pt}
\titlespacing\subsection{0pt}{6pt plus 4pt minus 2pt}{0pt plus 2pt minus 2pt}
\titlespacing\subsubsection{0pt}{6pt plus 4pt minus 2pt}{0pt plus 2pt minus 2pt}

\titleformat*{\section}{\large\bfseries}
\titleformat*{\subsection}{\normalsize\bfseries}
\titleformat*{\subsubsection}{\normalsize\bfseries}
\let\oldfootnotesize\footnotesize
\renewcommand{\footnotesize}{\fontsize{8bp}{1em}\selectfont}
\renewcommand{\cite}{\autocite} % citations in footnotes
\bibliography{biblio}

\headheight=14pt

\hypersetup{
    pdftitle={H2020-MSCA-IF-2014},    % title
    pdfauthor={Victor Gabillon},
    colorlinks=true,
    citecolor=black,
    linkcolor=black,
    urlcolor=blue
  }

\pagestyle{fancy}
\fancyhead{}
\fancyhead[C]{\acronym---Standard EF}
\fancyfoot{}
\fancyfoot[C]{Part B - Page \thepage~of \pageref{LastPage}}


\definecolor{resRouge}{RGB}{3,35,48}
\definecolor{resCas}{RGB}{255,54,35}
\definecolor{resNoir1}{RGB}{36,75,107}
\definecolor{resNoir2}{RGB}{255,54,35}
\definecolor{resNoir3}{RGB}{3,35,48}




\renewcommand{\headrulewidth}{0pt}


\renewcommand{\contentsname}{TABLE OF CONTENTS}


\newdimen\pHeight
\pHeight=-32678sp
\newdimen\pLower
\pLower=-4096sp
\newdimen\pLineWidth
\pLineWidth=32678sp
\newdimen\pKern
\pKern=-276480sp
\newdimen\pIR
\pIR=-131072sp
\newsavebox{\Cbox}
\newcommand{\HRule}{\rule{\linewidth}{0.5mm}}

\newsavebox{\vertCmplx}
\newdimen\Cheight
\newdimen\Cwidth
\sbox{\Cbox}{\rm C}
\Cheight=\ht\Cbox
\Cwidth=\wd\Cbox
\advance\Cheight by \pHeight
\sbox{\vertCmplx}{\rule[\pLower]{\pLineWidth}{\Cheight}}
\sbox{\Cbox}{\usebox{\Cbox}\kern\pKern\usebox{\vertCmplx}}
\wd\Cbox=\Cwidth
%\def\C{\usebox{\Cbox}}
%\def\Re{{\rm I\kern\pIR R}}
\def\Re{\mathbb{R}}
\def\Nat{{\rm I\kern\pIR N}}
%\def\argmax{\mathop{\rm arg\,max}}
%\def\argmin{\mathop{\rm arg\,min}}
\def\liminf{\mathop{\rm lim\,inf}}
\def\log{\mathop{\rm log}}
\def\Mean{{\bf Mean}}
\def\Var{{\bf Var}}
\def\Cov{{\bf Cov}}
\def\exptE{{\E}}
%\def\diag{\mathop{\rm diag}}

\newcommand{\bsmall}[1]{\vspace{#1}\begin{small}}
\newcommand{\esmall}[1]{\end{small}\vspace{#1}}



\newcommand{\argmax}{\operatorname*{argmax}} %\operatorname* pour les op. pouvant admettre des limites...
\newcommand{\argmin}{\operatorname*{argmin}}

\newcommand{\diag}{\operatorname*{diag}}

\newcommand{\greedy}{\operatorname*{{\cal G}}}
\newcommand{\lip}{\operatorname*{Lip}}
\renewcommand{\mod}{\operatorname*{mod}}

\newcommand{\bruno}[1]{~\\
{\bf[Comment (Bruno)]: \emph{#1}}
~\\
}

\newcommand\pp[1]{\Gamma^{#1}}

\def\defi{\stackrel{\Delta}{=}}
\def\E{\mathbb{E}}
\def\R{\mathbb{R}}
\def\G{{\cal G}}
\def\F{{\cal F}}
\def\bv{{\bf v}}
\def\bq{{\bf q}}
\def\bpi{\boldsymbol{\pi}}


\def\proj{{\cal A}}

\def\ie{{that is} }
\def\eg{{for instance} }
\def\cf{{see} }


\newcommand\mpi[1]{$\mbox{AMPI}_{#1}$}
\newcommand\cbmpi[1]{$\mbox{CBMPI}_{#1}$}



\def\A{{\mathcal{A}}}
\def\B{{\mathcal{B}}}
\def\C{{\mathcal{C}}}
\def\D{{\mathcal{D}}}
\def\E{{\mathbb{E}}}
\def\F{{\mathcal{F}}}
\def\G{{\mathcal{G}}}
%\def\H{{\mathcal{H}}}

\def\K{{\mathcal{K}}}
\def\I{{\mathcal{I}}}
\def\J{{\mathcal{J}}}
\def\L{{\mathcal{L}}}
\def\M{{\mathcal{M}}}
\def\N{{\mathcal{N}}}
\def\O{{\mathcal{O}}}
\def\P{{\mathcal{P}}}
\def\Q{{\mathcal{Q}}}
%\def\R{{\mathcal{R}}}
\def\S{{\mathcal{S}}}
\def\T{{\mathcal{T}}}
\def\U{{\mathcal{U}}}
\def\V{{\mathcal{V}}}
\def\W{{\mathcal{W}}}
\def\X{{\mathcal{X}}}
\def\Z{{\mathcal{Z}}}

\def\vecalpha{{\boldsymbol{\alpha}}}
\def\vecmu{{\boldsymbol{\mu}}}
\def\vectheta{{\boldsymbol{\theta}}}
\def\vecpi{{\boldsymbol{\pi}}}
\def\vecpsi{{\boldsymbol{\psi}}}
\def\vecphi{{\boldsymbol{\phi}}}
\def\vec0{{\boldsymbol{0}}}
\def\vecf{{\boldsymbol{f}}}
\def\veck{{\boldsymbol{k}}}
\def\vecp{{\boldsymbol{p}}}
\def\vecq{{\boldsymbol{q}}}
\def\vecu{{\boldsymbol{u}}}
\def\vecv{{\boldsymbol{v}}}
\def\vecw{{\boldsymbol{w}}}
\def\vecx{{\boldsymbol{x}}}
\def\vecy{{\boldsymbol{y}}}
\def\vecz{{\boldsymbol{z}}}

\def\matSigma{{\boldsymbol{\Sigma}}}
\def\matA{{\boldsymbol{A}}}
\def\matC{{\boldsymbol{C}}}
\def\matF{{\boldsymbol{F}}}
\def\matG{{\boldsymbol{G}}}
\def\matH{{\boldsymbol{H}}}
\def\matI{{\boldsymbol{I}}}
\def\matK{{\boldsymbol{K}}}
\def\matP{{\boldsymbol{P}}}
\def\matU{{\boldsymbol{U}}}
\def\matY{{\boldsymbol{Y}}}
\def\matZ{{\boldsymbol{Z}}}

\def\setP{{\mathbb{P}}}

\newcommand{\iid}{\stackrel{iid}{\sim}}

% Symbols
\newcommand{\alg}{\mathcal A}
\renewcommand{\Re}{\mathbb R}
%\def\argmax{\mathop{\rm arg\,max}}
%\def\argmin{\mathop{\rm arg\,min}}
\def\arginf{\mathop{\rm arg\,inf}}
% MDP notation
\newcommand{\MDP}{\mathcal M}
\newcommand{\discount}{\gamma}
\newcommand{\state}{\S}
\newcommand{\action}{\mathcal{A}}
\newcommand{\reward}{r}
\newcommand{\Return}{R}
%\newcommand{\truncReturn}{\overline{R}}
\newcommand{\truncReturn}{R}
\newcommand{\VFReturn}{\widetilde{R}}
\newcommand{\dynamics}{p}
%\newcommand{\reward}{\mathcal{R}}
%\newcommand{\dynamics}{\mathcal{P}}

\newcommand{\rolSetDistri}{\mu}

\newcommand{\RolloutSize}{m}

\newcommand{\pol}{\pi}
\newcommand{\polSpace}{\Pi}
\newcommand{\polDist}{\delta}
\newcommand{\fun}{f}
\newcommand{\Data}{\mathcal{D}}
\newcommand{\funSpace}{\mathcal{F}}
\newcommand{\vSpace}{\mathcal{B}^v}
\newcommand{\qSpace}{\mathcal{B}^Q}
\newcommand{\vPiSpace}{\mathcal{B}^{\pol}}
%\newcommand{\pseudoDim}{\mathcal{V}^+}
\newcommand{\pseudoDim}[1]{V_{#1^+}}
\newcommand{\cover}{\mathcal{N}}

\newcommand{\Rmax}{R_{\max}}

\newcommand{\Vfun}{v}
%\newcommand{\Vmax}{V_{\max}}
\newcommand{\Vmax}{V_{\max}}
\newcommand{\hV}{\widehat{\Vfun}}
\newcommand{\hv}{\widehat{\Vfun}}
\newcommand{\bV}{\overline{\Vfun}}

\newcommand{\Qfun}{Q}
\newcommand{\estQfun}{\widehat{\Qfun}}
\newcommand{\Qmax}{Q_{\max}}
%\newcommand{\Qmax}{q}
\newcommand{\hQ}{\widehat{Q}}

\newcommand{\hmu}{\widehat{\rho}}
\newcommand{\hrho}{\widehat{\mu}}
\newcommand{\hell}{\widehat{\ell}}

\newcommand{\hhmu}{\widehat{\mu}}

\newcommand{\greedyPol}{\mathcal G}
\newcommand{\data}{\mathcal D}
%\newcommand{\ibar}{\bar{i}}
\newcommand{\ibar}{j}

\newcommand{\ind}[1]{\mathbb I\left\lbrace {#1} \right\rbrace}

\newcommand{\norm}[1]{\Arrowvert{#1}\Arrowvert}
\newcommand{\normMu}[1]{||{#1}||_{1,\mu}}
\newcommand{\normNu}[1]{||{#1}||_{1,\nu}}
\newcommand{\err}{\varepsilon}
%\newcommand{\expErr}{\ell}
%\newcommand{\empErr}{\widehat{\ell}}
\newcommand{\expErr}{\epsilon'}
\newcommand{\empErr}{\widehat{\epsilon}'}

\newcommand{\lateDist}{{\rho}}
\newcommand{\initDist}{{\mu}}
\newcommand{\nSamplesDPI}{{N'}}
\newcommand{\sumSamplesDPI}{\sum_{i=1}^{\nSamplesDPI}}
\newcommand{\nSamples}{{N'}}
\newcommand{\sumSamples}{\sum_{i=1}^{\nSamples}}
\newcommand{\nRolls}{M}
\newcommand{\sumRolls}{\sum_{j=1}^{\nRolls}}
\newcommand{\avgActions}{\frac{1}{|\action|}\sum_{a\in\action}}
\newcommand{\sumActions}{\sum_{a\in\action}}

\newcommand{\probParam}{\delta}
% Indecies, number of elements, summations, products, and ranges
\newcommand{\idxTask}{m}
\newcommand{\idxTaskB}{m'}
\newcommand{\idxTaskNew}{\nTasks+1}
\newcommand{\idxSample}{n}
\newcommand{\idxST}{mn}
\newcommand{\idxClass}{c}
\newcommand{\nTasks}{M}
\newcommand{\nTasksClass}{\nTasks_{\class}}
\newcommand{\nTasksClassBut}{\nTasks_{-\idxTask,\class}}
\newcommand{\nReplica}{{N_{r}}}
\newcommand{\nClasses}{C}
\newcommand{\sumTask}{\sum_{\idxTask=1}^{\nTasks}}
\newcommand{\sumSample}{\sum_{\idxSample=1}^{\nSamples}}
\newcommand{\sumClass}{\sum_{\idxClass=1}^{\nClasses}}
\newcommand{\prodTask}{\prod_{\idxTask=1}^{\nTasks}}
\newcommand{\prodSample}{\prod_{\idxSample=1}^{\nSamples}}
\newcommand{\prodClass}{\prod_{\idxClass=1}^{\nClasses}}
\newcommand{\rangeTask}{\idxTask=1,\ldots,\nTasks}
\newcommand{\rangeSample}{\idxSample=1,\ldots,\nSamples}
\newcommand{\rangeClass}{\idxClass=1,\ldots,\nClasses}
\newcommand{\rangeClassInf}{\idxClass=1,\ldots,\infty}

\newcommand{\iter}{t}
\newcommand{\class}{c}
\newcommand{\classVectIter}{\textbf{c}^{(\iter)}}
\newcommand{\classVect}{\textbf{c}}
\newcommand{\classTask}{\class_{\idxTask}}
\newcommand{\classTaskB}{\class_{\idxTask'}}
\newcommand{\classTaskNew}{\class_{\idxTaskNew}}
\newcommand{\meanClass}{\mean_{\class}}
\newcommand{\precClass}{\precis_{\class}}
\newcommand{\covClass}{\cov_{\class}}
\newcommand{\varClass}{\var_{\class}}
\newcommand{\meanClassTask}{\mean_{\classTask}}
\newcommand{\precClassTask}{\precis_{\classTask}}
\newcommand{\covClassTask}{\cov_{\classTask}}
\newcommand{\varClassTask}{\var_{\classTask}}

\newcommand{\vfun}{V}
\newcommand{\vfunTask}{\vfun_{\idxTask}}
\newcommand{\vfunTMean}{\mean_{\vfunT}}
\newcommand{\vfunTCov}{\cov_{\vfunT}}
\newcommand{\vfunTMeanPost}{\meanPost_{\vfunT}}
\newcommand{\vfunTCovPost}{\covPost_{\vfunT}}

\newcommand{\xtest}{x_*}

\newcommand{\w}{{\bold w}}
\newcommand{\wAvg}{\bar{\w}}
\newcommand{\wFake}{\widehat{\w}}
\newcommand{\wTask}{\w_{\idxTask}}
\newcommand{\wTaskB}{\w_{\idxTask'}}
\newcommand{\wTaskNew}{\w_{\idxTaskNew}}
\newcommand{\wFakeTask}{\wFake_{\idxTask}}
\newcommand{\wFakeTaskNew}{\wFake_{\idxTaskNew}}
\newcommand{\wFakeTaskB}{\wFake_{\idxTask'}}
\newcommand{\wTaskMean}{\mean_{\idxTask}}
\newcommand{\wTaskCov}{\cov_{\idxTask}}
\newcommand{\wTaskPrec}{\precis_{\idxTask}}
\newcommand{\wTaskMeanPost}{\meanPost_{\idxTask}}
\newcommand{\wTaskPrecPost}{\precPost_{\idxTask}}
\newcommand{\wTaskCovPost}{\covPost_{\idxTask}}


\newcommand{\latent}{\mathcal Z}

\newcommand{\coeff}{\alpha}
\newcommand{\coeffT}{\coeff_{\task}}
\newcommand{\coeffTMean}{\mean_{\coeff}}
\newcommand{\coeffTCov}{\cov_{\coeff}}
\newcommand{\coeffTMeanPost}{\meanPost_{\coeffT}}
\newcommand{\coeffTCovPost}{\covPost_{\coeffT}}

\newcommand{\nfeatures}{s}
\newcommand{\dims}{d}

\newcommand{\featFun}{\phi}
\newcommand{\featVec}{\boldsymbol{\phi}}
\newcommand{\featMat}{\boldsymbol{\Phi}}
\newcommand{\featMatTask}{\featMat_{\idxTask}}
\newcommand{\featMatST}{\featMat_{\idxST}}

\newcommand{\kernelFun}{k}
\newcommand{\kernelFunT}{k_{\task}}
\newcommand{\kernelMat}{K}
\newcommand{\kernelMatT}{\kernelMat_{\task}}

\newcommand{\inSpace}{\mathcal X}
\newcommand{\inElemTask}{X_{\idxTask}}
\newcommand{\inElem}{x}
\newcommand{\inElemST}{\inElem_{\idxST}}
\newcommand{\inElemSTB}{\inElem'_{\idxST}}

\newcommand{\obs}{r}
\newcommand{\obsTask}{R_{\idxTask}}
\newcommand{\obsTaskB}{R_{\idxTask'}}
\newcommand{\obsTaskNew}{R_{\idxTaskNew}}

\newcommand{\outT}{R_{\idxTask}}
\newcommand{\outElemT}{r^i_\idxTask}
\newcommand{\outElem}[2]{r_{#2}^{#1}}

\newcommand{\noise}{N}
\newcommand{\noiseT}{\noise_{\task}}
\newcommand{\noiseTVar}{\Sigma_{\task}}
\newcommand{\noiseTVarB}{\var\transfT\transpose{\transfT}}
\newcommand{\stddev}{\sigma}

% Partial equations
\newcommand{\featHTTask}{\featMatTask\transpose{\matH}}
\newcommand{\HfeatTTask}{\matH\transpose{\featMat}_{\idxTask}}
\newcommand{\HfeatTTaskSample}{\matH\transpose{\featMat}_{\idxST}}
\newcommand{\invHHT}{\invA{{\matH\transpose{\matH}}}}
\newcommand{\HHT}{\matH\transpose{\matH}}

% Distributions and parameters
\newcommand{\normal}{\mathcal N}
\newcommand{\mean}{\boldsymbol{\mu}}
\newcommand{\meanPrior}{\mean_0}
\newcommand{\meanPost}{{\meanPrior'}}
\newcommand{\meanPriorRep}{\widetilde{\mean}_0}
\newcommand{\meanPostRep}{{\meanPriorRep'}}
\newcommand{\meanClassPrior}{\mean_{0\idxClass}}
\newcommand{\meanClassPriorRep}{\widetilde{\mean}_{0\idxClass}}
\newcommand{\meanClassPostRep}{{\meanClassPriorRep'}}

\newcommand{\var}{\sigma^2}
\newcommand{\varPrior}{\var_0}
\newcommand{\varPost}{\varPrior'}
\newcommand{\varPriorRep}{\widetilde{\var}_0}
\newcommand{\varPostRep}{\varPriorRep'}
\newcommand{\varClassPrior}{\var_{0\idxClass}}
\newcommand{\varClassPriorRep}{\widetilde{\var}_{0\idxClass}}
\newcommand{\varClassPostRep}{\varClassPriorRep'}

\newcommand{\cov}{\mathbf{\Sigma}}
\newcommand{\covB}{\mathbf{\Sigma}'}
\newcommand{\covPrior}{\cov_0}
\newcommand{\covPost}{\covPrior'}
\newcommand{\covPriorRep}{\widetilde{\cov}_0}
\newcommand{\covPostRep}{\covPriorRep'}
\newcommand{\covClassPrior}{\cov_{0\idxClass}}
\newcommand{\covClassPriorRep}{\widetilde{\cov}_{0\idxClass}}
\newcommand{\covClassPostRep}{\covClassPriorRep'}

\newcommand{\precis}{\Lambda}
\newcommand{\precPrior}{\precis_0}
\newcommand{\precPost}{\precisPrior'}
\newcommand{\precPriorRep}{\widetilde{\precis}_0}
\newcommand{\precPostRep}{\precisPriorRep'}
\newcommand{\precClassPrior}{\precis_{0\idxClass}}
\newcommand{\precClassPriorRep}{\widetilde{\precis}_{0\idxClass}}
\newcommand{\precClassPostRep}{\precisClassPriorRep'}

\newcommand{\iwishart}{\mathcal {IW}}
\newcommand{\wishart}{\mathcal W}

\newcommand{\wishScale}{k}
\newcommand{\wishScalePrior}{\wishScale_0}
\newcommand{\wishScalePost}{\wishScalePrior'}
\newcommand{\wishScalePriorRep}{\widetilde{\wishScale}_0}
\newcommand{\wishScalePostRep}{\wishScalePriorRep'}
\newcommand{\wishScaleClassPrior}{\wishScale_{0\idxClass}}
\newcommand{\wishScaleClassPriorRep}{\widetilde{\wishScale}_{0\idxClass}}
\newcommand{\wishScaleClassPostRep}{\wishScaleClassPriorRep'}

\newcommand{\freedom}{\nu}
\newcommand{\freedomPrior}{\freedom_0}
\newcommand{\freedomPost}{\freedomPrior'}
\newcommand{\freedomPriorRep}{\widetilde{\freedom}_0}
\newcommand{\freedomPostRep}{\freedomPriorRep'}
\newcommand{\freedomClassPrior}{\freedom_{0\idxClass}}
\newcommand{\freedomClassPriorRep}{\widetilde{\freedom}_{0\idxClass}}
\newcommand{\freedomClassPostRep}{\freedomClassPriorRep'}

\newcommand{\covNoise}{\mathbf{S}}

\newcommand{\igamma}{\mathcal {IG}}
\newcommand{\igScale}{\beta}
\newcommand{\igScalePrior}{\igScale_0}
\newcommand{\igScalePost}{\igScalePrior'}
\newcommand{\igScalePriorRep}{\widetilde{\igScale}_0}
\newcommand{\igScalePostRep}{\igScalePriorRep'}
\newcommand{\igScaleClassPrior}{\igScale_{0\idxClass}}
\newcommand{\igScaleClassPriorRep}{\widetilde{\igScale}_{0\idxClass}}
\newcommand{\igScaleClassPostRep}{\igScaleClassPriorRep'}

\newcommand{\igShape}{\alpha}
\newcommand{\igShapePrior}{\igShape_0}
\newcommand{\igShapePost}{\igShapePrior'}
\newcommand{\igShapePriorRep}{\widetilde{\igShape}_0}
\newcommand{\igShapePostRep}{\igShapePriorRep'}
\newcommand{\igShapeClassPrior}{\igShape_{0\idxClass}}
\newcommand{\igShapeClassPriorRep}{\widetilde{\igShape}_{0\idxClass}}
\newcommand{\igShapeClassPostRep}{\igShapeClassPriorRep'}

\newcommand{\mnomial}{\mathcal M}
\newcommand{\dirac}{\delta}

% Dirichlet process parameters and symbols
\newcommand{\baseDistro}{G_0}
\newcommand{\baseDistroPost}{\widetilde{G}}
\newcommand{\concentration}{\tau}
\newcommand{\stick}{v}
\newcommand{\stickClass}{\stick_\idxClass}
\newcommand{\mnCoeff}{\pi}
\newcommand{\mnCoeffClass}{\mnCoeff_\idxClass}

\newcommand{\param}{\theta}
\newcommand{\paramClass}{\param_\idxClass}
\newcommand{\paramClassTask}{\param_{\classTask}}
\newcommand{\hyper}{\psi}
\newcommand{\hyperPrior}{\hyper_0}
\newcommand{\hyperPriorRep}{\widetilde{\hyper}_0}
\newcommand{\hyperPost}{\hyperPrior'}
\newcommand{\hyperPostRep}{\hyperPriorRep}
\newcommand{\hyperClass}{\hyper_{0\idxClass}}
\newcommand{\hyperClassPost}{\hyperClass'}
\newcommand{\hyperClassRep}{\widetilde{\hyper}_{0\idxClass}}
\newcommand{\hyperClassPostRep}{\hyperClassRep'}

% Special matrices and operators

\newcommand{\matHT}{\transpose{\matH}}
\newcommand{\vectH}{h}
\newcommand{\identity}{I}
\newcommand{\identityA}[1]{I_{#1}}
\newcommand{\transpose}[1]{{#1}^\top}
\newcommand{\inv}[1]{#1^{-1}}
\newcommand{\invA}[1]{\left(#1\right)^{-1}}
\newcommand{\trace}[1]{\text{tr}\left(#1\right)}
\newcommand{\noisevec}{\mathcal E}

% Modified LQR parameters
\newcommand{\svar}{\eta^2}
\newcommand{\avar}{\delta^2}
\newcommand{\amean}{\lambda}

% Expectation, covariance and probability
\newcommand{\expectShort}{\mathbb E}
\newcommand{\expectA}[1]{\mathbb E \left[ {#1} \right]}
\newcommand{\expectB}[2]{\mathbb E_{#1} \left[ {#2} \right]}
\newcommand{\covA}[1]{\text{Cov} \left[ {#1} \right]}
\newcommand{\probA}[1]{\mathbb P\left[ {#1} \right]}
\newcommand{\probB}{p}

% Extras
\newcommand{\new}{\textrm{\scriptsize new}}
\newcommand{\paragTitle}[1]{\textbf{#1.}}
%\newcommand{\note}[1]{\textbf{Note.} \textit{#1}}
%\newcommand{\TODO}[1]{(\textbf{TODO: {#1}})}
\newcommand{\shrink}{\!\!\!}
%\newcommand{\eqref}[1]{(\ref{#1})}

%\newcommand{\errExpSTL}[1][]{er_{\task_{#1}}}
%\newcommand{\errExpSTLOpt}[1][]{er^*_{\task_{#1}}}
%\newcommand{\dotprod}[2]{\langle #1, #2 \rangle}

\newcommand{\LinSpace}{\F}
\newcommand{\TruncSpace}{\widetilde{\LinSpace}}
\newcommand{\distri}{\mu}
\newcommand{\empDistri}{\widehat{\distri}}

\newcommand{\q}{Q}

\newcommand{\appdx}[1]{Appendix~{#1}}
\newcommand{\eqrefb}[1]{Equation~\eqref{eq:#1}}


\def\AMPIV{\mbox{AMPI-V }}
\def\AMPIQ{\mbox{AMPI-Q }}


\def\szl{Szita and L{\H{o}}rincz }
\def\Ev{{\mathcal{E}}}



\begin{document}

\phantom{a}
\vspace{15mm}
\begin{center}


        \Large{
      
     
        \textbf{START PAGE}
  
          \vspace{15mm}
          MARIE SKLODOWSKA-CURIE ACTIONS\\
          \vspace{1cm}
          
          \textbf{Individual Fellowships (IF)}\\
          \textbf{Call: H2020-MSCA-IF-2015}
          \vspace{2cm}                   

          PART B
          \vspace{2.5cm}

          ``\acronym''
          \vspace{2cm}

          \textbf{This proposal is to be evaluated as:}
          \vspace{.5cm}

          \textbf{[Standard EF]}
        }

  \end{center}
\vspace{1cm}

\newpage
\setcounter{tocdepth}{1}
\setcounter{section}{-1}
\tableofcontents


\newpage
\section{List of Participants}
\label{sec:participants}

\newcommand\rotx[1]{\rotatebox[origin=c]{90}{\textbf{#1}}}
\newcommand\roty[1]{\rotatebox[origin=c]{90}{\parbox{4cm}{\raggedright\textbf{#1}}}}
\newcommand\MyHead[2]{\multicolumn{1}{l|}{\parbox{#1}{\centering #2}}}

\noindent\begin{tabular}{|m{2.4cm}|m{1.5cm}|b{1em}|b{1em}|m{2cm}|m{2.5cm}|m{2cm}|c|}
\hline
  \textbf{Participants}
& \MyHead{1cm}{\textbf{Legal\\Entity\\Short\\Name}}
& \rotx{Academic}
& \rotx{Non-academic}
& \textbf{Country}
& \MyHead{2.1cm}{\textbf{Dept. / \\Division / \\Laboratory}}
& \textbf{Supervisor}
& \MyHead{2.5cm}{\textbf{Role of\\Partner\\Organisation}} \\
\hline
\underline{Beneficiary} & & & & & & & \\\hline
Lancaster University  & ULANC & X & & United Kingdom & Department of Mathematics and Statistics & Prof David Leslie & \\\hline
%\underline{Partner} \underline{Organisation} & & & & & & & \\\hline
%- NAME  & & & & & & & \\\hline
\end{tabular}
\vspace{\baselineskip}

%Data for non-academic beneficiaries
%
%\noindent\begin{tabular}{|m{1.7cm}|m{2cm}|m{1.8cm}|c|c|m{2.5cm}|c|c|c|}
%\hline
%  \textbf{Name}
%& \roty{Location of research premises (city / country)}
%& \roty{Type of R\&D activities}
%& \roty{No. of fulltime employees}
%& \roty{No. of employees in R\&D}
%& \roty{Website}
%& \roty{Annual turnover (approx. in Euro)}
%& \roty{Enterprise status (Yes/No)}
%& \roty{SME status  (Yes/No)}
%\\\hline
%& & & & & & & & \\\hline
%\end{tabular}
%\vspace{\baselineskip}
%
%Note that:
%\begin{itemize}
%\item Any inter-relationship between different participating institutions or individuals (e.g. family ties, shared premises or facilities, joint ownership, financial interest, overlapping staff or directors, etc.) must be declared and justified in this part of the proposal;
%\item The information in the table for non-academic beneficiaries must be based on current data, not projections;
%\item The data provided relating to the capacity of the participating institutions will be subject to verification during the Grant Agreement preparation phase.
%\end{itemize}


\newpage

\section{Excellence}
\label{sec:excellence}



\subsection{Quality, innovative aspects and credibility of the research (including inter/multidisciplinary aspects)}
\label{sec:quality}

You should develop your proposal according to the following lines:
\begin{itemize}
\item Introduction, state-of-the-art, objectives and overview of the action
\item Research methodology and approach: highlight the type of research and innovation activities proposed
\item Originality and innovative aspects of the research programme: explain the contribution that the project is expected to make to advancements within the project field. Describe any novel concepts, approaches or methods that will be employed.
\end{itemize}
Explain how the high-quality, novel research is the most likely to open up the best career possibilities for the Experienced Researcher and new collaboration opportunities for the host organisation(s).


\TODO{Add a paragraph on general security.}

Security is an important aspect in modern world.
Effectively protecting the ports, airports, and other transportation systems from malicious attacks,
fighting the trafficking of drugs, and firearms, and securing proprietary and sensitive information over the ever-growing, modern cyber networks comprise some of the main axes of this critical task.
A common challenge in all these problems is that 

As a solid mathematical framework to model strategic decision making, game theory has proved useful in many real-world applications from economics and political science to logic, computer science and psychology. 
Security resource allocations and scheduling problems comprise yet another application area of critical concern, that has recently been shown to greatly benefit from game-theoretic approaches.
Since 2007, the so-called ARMOR software \cite{pita2008deployed} is used at the Los Angeles International Airport (LAX) to effectively determine 
checkpoints on the roadways leading to the airport, and to canine patrol routes within terminals. 
Similarly, such programs as IRIS \cite{tsai2009iris}, PROTECT \cite{shieh2012protect}, and TRUSTS \cite{yin2012trusts} are respectively being deployed at the US Federal Air Marshals, the US coast guard patrolling, and the Los Angeles Metro system's fare inspection strategy.

\TODO{Add much more about the importance and the impact of machine learning}
These methods, while being remarkably effective in their corresponding application arenas, usually rely on a pre-defined model of the environment. However, such information may in general not be available in many real-world scenarios. A key objective forming the basis of this grant proposal, is thus to design efficient and theoretically sound, data-driven methods that can actively interact with the environment to {\em learn} a fair model through repeated games. As discussed in the sequel, this may be achieved in an online fashion or through an exploration phase prior to the algorithm's final launch.  
introduce the need for sequential decision making under uncertainty and online learning




\noindent \textbf{\textit{\\State-of-the-art}}
\noindent \textbf{\textit{\\Game Theory meets Security}}\\
\TODO{better distinction between minimax nash stacjkelberg}
From a game-theoretic perspective, a security problem is viewed as a two-player game that captures the interaction between a defender (e.g., border patrols, metro inspectors, network administrators) and an attacker (e.g., terrorists/drug smugglers, illegal metro users, malicious cyber attackers). The action of the defender (attacker) is defined as selecting a subset of targets to protect (attack). For each defender/attacker action pair, utilities are defined as the players' gain or loss, and the players' objectives are to maximise their corresponding pay-offs. From the defender's perspective, this corresponds to efficiently allocating a limited number of resources to secure some predefined targets from the attacker. 
%The defender allocates a limited number of resources to secure some predefined target, which may under threat by the attacker. 
Solutions to such games rely on randomised strategies, making the defender's scheme highly unpredictable for the attacker, thus giving rise to a significant advantage over the original mechanisms that are based on deterministic human schedulers. In the case of games that are fully competitive between the two players  (i.e. the so-called zero-sum games), these methods are provably robust in that they provide guaranteed performance against {\em any} possible attacker. In this case, such guarantees hold, even if the defender's strategy is completely revealed to the attacker.  
The extension of this guarantee to a more general (non zero-sum) game is provided by Stackelberg equilibrium, a notion that generalises the famous Nash equilibrium \cite{korzhyk2011stackelberg}. 
%Beyond these fundamental results, 
%some questions forming the primary focus of research in security games have been scalability, robustness with respect to uncertainty on the players utilities, or devising strategies that take advantage of the attacker's potentially limited rationality or bounded memory\cite{tambe2012game}. 

\noindent \textbf{\textit{\\Sequential decision-making under uncertainty}}\\
Machine learning is a field of artificial intelligence where the goal is to design software able to extract information from data so that the machine itself can make use of this information to take  autonomous decisions.
The problem of sequential decision-making under uncertainty arises in everyday life, when we try to find an answer to questions like how to navigate from home to work, how to play and win a game (e.g.,~backgammon, poker, or the game of Tetris that has been used as an experimental testbed in Victor's thesis), how to retrieve our information of interest from the Internet, how to optimize the performance of a factory, etc. To solve this problems machine learning has proposes methods and techniques that borrow and extend a lot from the fields of statistics (in order to take into account the uncertainty around the data) and optimisation (to create fast converging methods). It has proven to make a difference in practice and has nice mathematical background. We want to focus here on two particular methods that are interesting in machine learning and where Victor and David are experts.
First, many interesting sequential decision-making tasks can be formulated as reinforcement learning (RL) problems. In RL, an agent interacts with a dynamic, stochastic, and incompletely known environment with the goal of learning a strategy or \textit{policy} to optimize some measure of its long-term performance (e.g.,~to remove as many lines as possible in Tetris).
talk very quickly about MDP POMDP tree search?

Sequential decision-making tasks is often coupled with online learning in the sense such problem needs to learn online how to solve a task while solving it. These two features, sequential decision making and online learning while be most required for our new problems right! elaborate a bit
The very deep and fundamental challenge that when using online learning comes with partial feedback  that try to solve a problems while simultaneously learning the parameters of the problem itself needs to address in order to be effective is the trade off that arises between the simultaneous needs to use on the solution we currently think is the best to because we need to sove the problem  (exploitation)  while also wanting to test possible other choice that might or might not be better (exploration)

Maybe add success of bandits.

A simple framework where the states are not involving but where the trade off between exploration and exploitation is there is the multi armed bandit problem.
The multi-armed bandit problem is a general framework with so many application where a 
The multi-armed bandit problem can be formalized as a game between a environment and a forecaster. At each round $t$, the forecaster pulls an arm $I(t)\in A$ and only observes a reward $l_{I(t),t}$.  Partial information games! In a security context you could imagine that the K arm/options are K possible security strategies whose value can  only assess their value by using them. There are different ways to measure the performance which will be of interest in this proposal.  The first formulation corresponds to the classical cumulative regret setting where the forecaster tries to constantly choose the security strategy with the highest value.  The second one is the ``pure exploration'' setting, where the forecaster uses the exploration phase to find a security solution according to which he will be evaluated at the end of this phase.

The cumulative regret setting is the standard formulation for multi-armed bandits. In this formulation, the objective is to minimize the expected cumulative regret defined as
%
\begin{equation*}
R(n)=\max_i\sum l_{i,t}-\E\left[\sum_{t=1}^n l_{I(t),t}\right].
\end{equation*}
%

The problem has been study under two main assumption 1) the environment chooses the reward iid according to some unknown distribution this setting is nice when when delaing with fphenomena that we know are of simple nature. a popular efficent  algorithm for it is On the other hand,~\mbox{\cite{Kaufmann12TS}} provided a distribution dependent analysis of the Thomson sampling algorithm, a Bayesian method which has been shown to be efficient in practice~\cite{Chapelle11EE}. 2)adversarial bandits 
\cite{Auer03NS}. This is much more general as it makes no assumption on the losses very robust

Another possible way to extent the initial formulation is to consider cases with an infinite number of arms.
The case where no structure is assumed on the set of arms has been considered by~\cite{Wang08AI} under a stochastic assumption. When the bandits are in a linear structure,~\cite{Abbasi-Yadkori11IA} tackled the case of stochastic rewards while~\cite{Dani07TP} tackled the adversarial case. 

The pure exploration is a relatively new setting, where the forecaster is only evaluated at the end of an exploration phase. Contrary to the cumulative regret setting, the rewards collected before the end of the game are not taken into account. 
The objective is  play between the probability of eroor and the smaples reqiuired. 
In the \textit{fixed confidence} setting (see e.g.,~\cite{Maron93HR,Even-Dar06AE}), the forecaster tries to minimize the number of rounds needed to achieve a fixed confidence on the quality of the returned best arm(s) while in the \textit{fixed budget} setting (see e.g.,~\cite{Bubeck09PE,Audibert10BA}), the number of rounds of the exploration phase is fixed and is known by the forecaster, and the objective is to maximize the probability of returning the best arm(s). 

The problems put forward a notion of complexity of the problem. That we try to characterise well. And that we will try to study in our new problems.

\noindent \textbf{\textit{\\Frontiers of Security Games: From handling uncertainty towards self-learning algorithms}}\\
\textbf{Related Work.}
Some of the main issues forming the primary focus of research in security games have been scalability, or devising strategies that take advantage of the attacker's potentially limited rationality or bounded memory\cite{tambe2012game}. Another important research goal that has been extensively addressed is to devise methods that are robust with respect to uncertainty about the environment\cite{Nguyen14RO, aghassi2006robust}.
However, little has been done to generalise the framework to a more realistic setting where the player's objective includes to actively learn the unknown environment. Achieving this goal is indeed crucial, since algorithms that make use of environmental knowledge are arguably more reliable than those merely designed to be robust against this lack of information. With this motivation, some interesting advancements have recently been made through links with optimisation and machine learning methods. 
These methods focus mostly on the case where the attacker's preferences are not fully known and are thus to be learned; the learning objective is achieved through a repeated a game.  \cite{blum2014learning, letchford2009learning} propose analyses in terms of the number of required queries to learn the optimal defender's strategy. 
\cite{Marecki12PR, qian2014online} take a Bayesian approach where, given a prior distribution, planning techniques based on Partially Observable Markov Decision Processes (POMDPs) are used to update the posterior over the adversary's preferences.
The main theoretical drawback of this planning method is in that the algorithm is based on Upper Confidence Trees (UCT), which, as shown by  \cite{munos2014bandits}, are provably sub-optimal. 
Recently an extended analysis is given by \cite{Balcan15CR}  for the case of multiple attackers, where at each round of the game, a single attacker is chosen adversarially from a fixed, finite, set of known attackers. The latter work shows strong connections with adversarial bandit theory. 

%\textbf{Motivation.} %%%V what can I bring


\textbf{Main Goal.} The purpose of this proposal is to solve real world security game problems that needs to be made un
to apply them to security games in a broad range of real world situations.
Our goal is to have a theoretically sound approach by designing efficient algorithms for which we can provide finite sample analysis.
Stochastic assumptions will be made when dealing with noise in the model and adversarial assumption when dealing with the adversary to make our approach both realistic and robust.
%%%A depending on the considered scenraio, it may make sense to consider the problem as adversarial or stochastic. The distinction will be made clear in the ... 

Give an example here of why you want to look at that!
One difference that we want to explore is that, contrary to the previously mentioned approaches, where the uncertainty is  about the attackers' utilities, we will explore the case where the uncertainty is on the utilities of the defender. This for instance happens when we can not assess for sure the precise return of a given action (checkpoint might not stop deterministically the attacks and the probability of success needs to be determined, here learned).



% (based on POMDPs and bandits) \cite{Blum15LP, Balcan15CR, qian2014online}.


We can also look to different formulations of the games that corresponds to real world possibilities or requirements:
Issues
\begin{itemize}
\item \textit{Efficient learning algorithms}
\item \textit{Scalable learning algorithms} Extending the previous approaches to complex problem that involves some combinatorial structure is also important. Submodularity
\item \textit{Robust learning:} Ensuring security in problems in problems suffering from uncertainty creates a very novel and exiting challenge as the uncertainty brings a new source of unsecurity. We will explore three ways to resplond to that.
A first way to  reply to that is to use setting where no assumption use of \textit{adversarial} algorithm.
 Another possible concern that might happen in some problems pure best arm identification 
 Another possibility is that in some situation we do not want the learning process to happen during the use of the program but before hand. Then we can assume that we use of a pre launch exploration phase where we try to learned as precisely as possible the model given some budget constraint or some targeted performance guarantees. 
  
 An
this need for robustness needs to be mitigated depending on the application. some real world problem might need an extra care on robustness like terrorist attack while others are not that sensitive to it.
 In the latter case we should not be to conservative in the learning to be able to learn faster.

 we might be required to learn defence strategies that are not necessarily the best in expectation but instead also guarantee not to possess large variances in their performance. Here we plan to make connection with risk averse learning algorithm.
\end{itemize}


\textbf{Objective 1 Pure exploration in Stackelberg games}
Bringing the pure exploration in bandits to Stacklerberg games is the natural first phase of our project. As it is first one of the main domain of expertise of Victor and two it brings a first conservative way to address learning without bringing to much risks for the learner. Indeed in this setting the learning of the unknown model happens during an exploration phase before putting on the market.  This for instance means that he can run tests of the security in a variety of predetermined attack scenario and therefore  probe his own probability of defence.

We would make the stochastic assumption as here it accounts for noise in our model and not adversary actions.
 The objective of this  approach is to determine  the best strategy during a given exploration phase and  is therefore  closely related to the general theory of optimisation and has been study in the discrete context of multi arm bandit as pure exploration problems \cite{Audibert10BA}. This initial work has been extended in a flurry variant setting where one tries to find the best(s) arms.
Victor has a nice expertise in that and has participated to the extension and application of such a framework in more and more complex problem (cite my work?) and is working on extension to combinatorial bandits that would improve upon the seminal work by Chen.


 Taking into account the particular  structure of the problem will be necessary when dealing with Stackelberg equilibrium in security games. There the function to optimise is even more complex. What is the complexity here?
 
 \paragraph{\textbullet$\;$  Complexity:} The hardness of the best arm identification problem in the stochastic setting can be interpreted as the total number of pulls required to discriminate the best arm(s) from the others. In simple multi-arm bandit setting it is defined as the sum of the complexity of each suboptimal options, where the complexity of a suboptimal option $i$ is invertionally proportional to the gap $\Delta_i= \mu^*- \mu_i$  the difference between the value of the the best option $\mu^*$ and the value of option $i$
 
 More precisely the complexity $H$ is  defined as
%
\begin{equation}
H = \sum_{k} \frac{1}{\Delta_k^2},
\end{equation}
%
Extensions of this complexity notion have been designed in more complex setting like combinatorial bandits cite{Chen}. Note that Victor is currently working on a improved version of this result. In a combinatorial setting, the forecaster must make a recommendation that is of combinatorial structure
 The (combinatorial) decision set is $\C\subseteq 2^K$ is such that any decision $U \in \C$ is a set of arms $U\subseteq \K$ and its value is the sum of their values, $\mu_U = \sum_{i \in U} \mu_i$. The value gap between two decisions is denoted by $\Delta_{U,V} = \mu_U - \mu_V$ and $U^* = \arg\max_{U\in\C} \mu_U$ is the best decision 
 \begin{align*}
\Delta^\odot_k = 
\begin{cases} 
\mu^* - \max\limits_{V\in\C: k\in V} \mu_V & \text{ if } k\notin V^* \\
\mu^* - \max\limits_{V\in\C: k\notin V} \mu_V & \text{ if } k\in V^* \\
\end{cases}
\end{align*}

In case of of games the picture would be even more complex as the complexity would depend on the actions that adverasry have available.

One first step is to relax the problem as shown in Krause et al finding the best response to a given adversary. This is known to be is NP hard problem  but can be solve almost optimally be a greedy algorithm thank to a sub modularity property of the problem. This gives rise to a first objective which would be to learning optimise stochastic submodular function under a pure exploration setting.
Note that I worked on similar subject with learning in submodular functions.

connections with risk averse (Cite the work of Amir Sani) maybe a separate section for this.
talk about the classical cumulative regret setting also!

\textbf{Objective 2 Learning  more complex adversarially chosen attacker in  Stakleberg}
The idea would be to  extend the work of Balcan using more complex bandit algorithms. They use a version with k known attackers. We can assume that k is extremely large but there is some  structure that permits us to use for instance combinatorial bandits.



\textbf{Objective 3 Repeated Network Security Games}
The security issue naturally has application in graph problem that model the network of roads/ connection between computers that agents might need to secure. Therefore there has been study that apply game theory to this problems. For instance it has been used to monitor road barrage in mumbai (connection) The goal is there to put some check point on a road to stop some terrorist. Its a one shot game where you try to minimise the probability of the player to pass.  Utilities are not really defined and complex here You just want to maximise the probability of catching the attacker. We are interested in a version of this game that is repeated . Everyday the same problem arises. We would minimise the cumulative regret. Therefore the defender can be adaptive and if the attacker is not smart and repeat always the same plan we will catch him often (not totally a worse case scenario). This can be seen actually has a specific problem of adversarial combinatorial bandits where the  attacker is limited to a very specific structure of losses which are path in a graph. We can expect to use the specificity of the graph by using some result from spectral graph theory. Maybe also we can use this theory to solve some issue with the scalability of the algorithm.



\noindent \textbf{\textit{\\Originality and innovative aspects of the research programme:}}\\

 
\noindent \textbf{\textit{\\timeliness and relevance:}}\\
Security is booming since 5 years
mahcine laerining also
the conjuntion of the two is definietly relevant but still largely unexplored
Europe wants security
migrants/ spy
 



%Or say that the approach that do only rarely propose strong theoretical results (to the notable exception of Blum)
%like \cite{Marecki12PR} (which by the way use a theoretically not sound algorithm UCT) 
\subsection{Clarity and quality of transfer of knowledge/training for the development of the researcher in light of the research objectives}
\label{sec:transfer}


The overall training objective is to significantly develop Dr Gabillon's scientific, organisational, communication and technology transfer skills.  This will enable him to continue building his portfolio of outstanding research to attain a position of independence and gain recognition in the international research community.

The proposed project is primarily a research project, and the main training objectives are to enhance the Dr Gabillon's scientific skills. Dr Gabillon is already an expert in the modern theory of bandits, including best arm identification, and reinforcement learning.  Therefore this project's main training objective will be to develop his skills and knowledge in statistical learning methods and game theory.  Prof.\ Leslie is an expert in both areas, and will of course assist the development of Dr Gabillon.  Further expertise in Lancaster from whom Dr Gabillon will learn includes the Statistical Learning group, in which he will be based, and the broader Statistics Research Group.

Combining the Department of Mathematics and Statistics with the Operations Research group within the Management School, Lancaster is the leading UK institution in bandit theory, with expertise in index policies (Glazebrook, Kirkbride, Jacko), Thompson sampling and contextual bandits (Grunewalder, Leslie) and application in medical trials (Vilar). Dr Gabillon will have ample opportunity to further develop his expertise in this area, and indeed brings expertise from a complementary aspect of online learning and decision-making in the design and analysis of algorithmic approaches to learning, especially with combinatorial bandit problems.   Dr Gabillon's expertise in best-arm identification will be of great interest to the Medical and Pharmaceutical Statistics research group.  He will present his research in this area to the research group and discuss possible applications in clinical trial design.  Furthermore his expertise in combinatorial bandits complements current industrially-funded research of the Prof.\ Leslie.

In addition to his research skills, Dr Gabillon will learn from the host's world-leading expertise in developing industrially-inspired statistics.  Statistical researchers in Lancaster have constant exposure to external companies, through the STOR-i Centre for Doctoral Training, and the Data Science Institute.  While embedded in this culture, Dr Gabillon will be given the opportunity to:
\begin{enumerate}
\item Gain further experience of developing industry/academic partnerships by working with Profs.\ Leslie and Eckley and other staff in STOR-i and the Data Science Institute in technology transfer activities.
\item Develop public communication skills by presenting research results to varied audiences.
\item Participate in the organisation of workshops in Lancaster and at the Royal Statistical Society.
\item Receive training on applying for funding by co-authoring proposals for UK and EU funding agencies.
\item Attend staff training workshops designed specifically for early-career researchers, and specifically the Research Development Programme, a structured development route for researchers, designed to promote impactful research and to support development beyond a disciplinary area.
\item Participate in teaching and research supervision (undergraduate and graduate).  This will not be obligatory, but Dr Gabillon will have the opportunity to benefit from peer observation and mentoring.\end{enumerate}

The UK Concordat to Support the Career Development of Researchers is an agreement between funders and employers of research staff to improve the employment and support for researchers and research careers in UK higher education.  Lancaster University is fully committed to the Concordat to Support the Career Development of Researchers and has put in place an Action Plan to support the full implementation of the Concordat at Lancaster.  Furthermore, throughout the fellowship, Dr Gabillon will adhere to the European Charter for Researchers, and the training objectives will be managed through a Personal Career Development Plan that Prof.\ Leslie and Dr Gabillon will write together.  This plan will be revised regularly throughout the fellowship to ensure that all objectives are met.  In addition, Dr Gabillon will have regular meetings with Prof.\ Leslie to discuss his research and to receive advice.

\subsection{Quality of the supervision and the hosting arrangements}
\label{sec:supervision}

%
%\subsubsection*{Qualifications and experience of the supervisor(s)}
%
%{\em
%Information regarding the supervisor(s) must include the level of experience on the research topic proposed and document its track record of work, including the main international collaborations. Information provided should include participation in projects, publications, patents and any other relevant results.
%To avoid duplication, the role and profile of the supervisor(s) should only be listed in the "Capacity of the Participating Organisations" tables (see section 6 below). 
%}
%
%\subsubsection*{Hosting arrangements\footnote{The hosting arrangements refer to the integration of the Researcher to his new environment in the premises of the Host. It does not refer to the infrastructure of the Host as described in Criterion Implementation.}}


%\fbox{\begin{minipage}{\textwidth}\paragraph{Career development}
%Employers and/or funders of researchers should draw up, preferably within the framework of their human resources management, a specific career development strategy for researchers at all stages of their career, regardless of their contractual situation, including for researchers on fixed-term contracts. It should include the availability of mentors involved in providing support and guidance for the personal and professional development of researchers, thus motivating them and contributing to reducing any insecurity in their professional future. All researchers should be made familiar with such provisions and arrangements.\end{minipage}}
%
%Therefore a Career Development Plan should not be included in the proposal, but it is part of implementing the project in line with the European Charter for Researchers.

\subsubsection*{Qualifications and experience of the supervisor(s)}

Prof.\ Leslie leads the Statistical Learning research group in the Department of Mathematics and Statistics, Lancaster University, and is Theme Lead for Foundations in Lancaster University's new Data Science Institute.  He is a world-leading researcher in statistical learning, Bayesian inference, decision-making and game theory, with 19 refereed articles in top journals of several different research fields, and collaborators from France, Singapore, USA and Australia.  His research on contextual bandit algorithms \cite{MayEtAl2012} is used by many of the world's largest companies to balance exploration and exploitation in real-time website optimisation.  He is expert in the mathematics of learning in games, \cite{LeslieCollins03,LeslieCollins05,LeslieCollins06,ChapmanEtAl2013,PerkinsLeslie2014} stochastic approximation, \cite{LeslieCollins03,PerkinsLeslie2012,PerkinsLeslie2014} and the mathematics of statistically-inspired reinforcement learning. \cite{LeslieCollins05,LarsenEtAl2010}  Prof.\ Leslie is the holder of a Google Faculty Award which funds a student to investigate multiple-action selection in bandits.  Prior to his relocation to Lancaster, he was a senior lecturer in the statistics group of the School of Mathematics, University of Bristol.  He continues to be co-director of the \pounds1.5m EPSRC-funded cross-disciplinary decision-making research group at the University of Bristol, and was on the management team of the \pounds5.5m ALADDIN project, a large strategic partnership between BAE Systems and EPSRC, involving researchers from Imperial College, Southampton, Oxford, Bristol and BAE Systems.

Prof.\ Leslie's mentoring approach is one of `guided freedom' in which the mentee takes responsibility for their own research, while regular discussions ensure that dead ends are avoided and promising openings are exploited.  In the 10 years since taking up a Faculty position, he has supervised 17 PhD students (5 now in Academic positions), 2 post-doctoral fellows, numerous MSc and undergraduate dissertations, and an undergraduate secondment from ENS Lyon.

\subsubsection*{Hosting arrangements}

The Researcher will be embedded within the statistical learning group which is lead by Prof.\ Leslie.  This is a team of 5 academic staff and around 5 PhD students within the Department of Mathematics and Statistics.  Dr Gabillon will participate in weekly group meetings and benefit from advice from the senior scientists in the group on research direction and management, personal development, workshop organisation, teaching, and other aspects of academic life.  The group also has extremely strong links with both the Data Science Institute (www.lancaster.ac.uk/dsi/) and the STOR-i Centre for Doctoral Training (www.stor-i.lancs.ac.uk/); each provides a weekly seminar series.  These exciting initiatives will provide multiple further opportunities to develop informal mentoring relationships in addition to the formal process which takes place for all staff at Lancaster University.  To ensure integration within these networks Dr Gabillon will be introduced to the groupings of researchers, invited to deliver a seminar on his research, and will participate in away days in which strong relationships are developed.


\subsection{Capacity of the researcher to reach and re-enforce a position of professional maturity in research}
\label{sec:maturity}
Professional maturity in academia would be ideally reached by leading a dynamic research group actively working on fundamental problems at the interface of game theory and online learning, with strong impact in real-world applications. Dr Gabillon has shown an extremely high potential to achieve this goal. As evident from his solid publication record, he has strong expertise in the domain, always giving equal importance to theory and applications. The fellow has also demonstrated strong ability to acquire new knowledge  and become highly productive in a short period of time. Indeed, a significant result of his PhD thesis is in bringing classical reinforcement learning algorithms closer to daily life. Moreover, aside from providing theoretical guarantees for his proposed methods, this entailed spending a significant amount of time single-handedly managing extensive parallel-computing experiments over a grid of computers, a task for which he had no prior knowledge. During the course of his PhD, through a 6-months internship at a major US R\&D lab (Technicolor Research Laboratory, Palo Alto), he had the opportunity to collaborate with a new team of R\&D researchers. He quickly became productive and his efforts in this short period of time have resulted in the publication of two peer-reviewed papers at prestigious international conferences in machine learning. Through this experience he has also obtained valuable knowledge about industrial research and its interaction with academia. This has given him the ability to better understand the research pathways to produce high-impact results and establish significant collaborations with industry.

At the start of the fellowship, Dr Gabillon will be closely mentored by Professor Leslie at Lancaster University. He will also have access to the university's research resources, and will be able to further develop his research and supervision skills, which will greatly contribute to achieving professional maturity. At Lancaster University, the fellow will also have the unique opportunity to establish inter-disciplinary collaborations through the recently established STOR-i program, a quality research training interface between statistics and industry. 

%[I PUT THE FOLLOWING TEXT IN HERE, WHICH DOESN'T SEEM TO HAVE MADE IT INTO Capacity.tex]
%
%Dr Gabillon's intention is to continue developing until he is able to build and lead a multi-disciplinary research group, developing innovative and impactful research in machine learning for decision-making, in collaboration with industry partners.  The initial stages of the researcher's career indicate that this is a realistic goal, with multiple high-quality publications in a short time frame, and already significant international research experience (see Section \ref{sec:cv}).  He has worked both as an academic researcher, and within a company \TODO{Which company, doing what?}, and the proposed research will broaden his research foundations significantly to allow further development as an independent researcher.  Working closely with Lancaster University's Data Science Institute, and STOR-i Centre for Doctoral Training, will not only develop a professional network of impact-aware academics, it will foster skills in the development of academia/industry partnerships.
%
%Furthermore, Dr Gabillon will establish a 2-year Personal Career Development Plan which he will update regularly under the mentorship of Prof.\ Leslie, the project Supervisor.  The Fellow will have access to Lancaster's central training resources, which will be used to further develop skills in teaching, public engagement, research supervision and management, which will contribute greatly to a position of professional maturity.  Moreover, by participating in weekly seminars, group meetings, and informal gatherings, the Fellow will have the opportunity to receive valuable feedback about his career development from the experience scientists in the Statistics Research Group.  This research group is visited by dozens of internationally-leading scientists each year, which will provide further opportunities to develop a research network and reach a position of professional maturity.
%

\section{Impact}
\label{sec:impact}

%\TODO{Demonstrate: worthwhile outreach, good communication strategy (are there existing connections that can be exploited?), adequate discussion of impact on researcher's career, indication of how outreach activities will be assessed, strategies for exploitation of outcomes.}

\subsection{Enhancing research- and innovation-related skills and working conditions to realise the potential of individuals and to provide new career perspectives}
\label{sec:enhancement}



Dr Gabillon is already a leading researcher in the mathematics of bandit algorithms and reinforcement learning.  This fellowship provides a training opportunity in two key additional research competences.  Firstly, he will develop an in depth knowledge of cutting edge statistical theory, and bring that to bear within bandit algorithms.  Training will be received from leading scientists in statistics and operations research at Lancaster University, and the many international visiting researchers who visit the department.   Secondly, Prof.\ Leslie is a leading expert on learning in games, as well as bandit algorithms, and will mentor Dr Gabillon to bring ideas from bandits into the game theoretical scenarios of this research proposal.  This significant broadening of the researcher's skill set will give him an extremely solid foundation on which to build a future research career.

In addition to pure research opportunities, Dr Gabillon will work within Lancaster University's extremely effective framework for industrial collaboration.  He will develop skills in how to manage the industry/academia relationship to ensure mutually beneficial outcomes.  This relationship-management will be a key skill for academics in the future; Lancaster University, and particularly the Department of Mathematics and Statistics, is currently a world-leading institution in developing such relationships.  Dr Gabillon will both be introduced to prospective industrial partners, and receive mentoring as he develops his own relationships.


\subsection{Effectiveness of the proposed measures for communication and results dissemination}

%{\em 
%\fbox{\begin{minipage}{\textwidth}
%\paragraph{Public engagement}
%Researchers should ensure that their research activities are made known to society at large in such a way that they can be understood by non-specialists, thereby improving the public's understanding of science. Direct engagement with the public will help researchers to better understand public interest in priorities for science and technology and also the public's concerns.
%\paragraph{Dissemination, exploitation of results}
%All researchers should ensure, in compliance with their contractual arrangements, that the results of their research are disseminated and exploited, e.g. communicated, transferred into other research settings or, if appropriate, commercialised. Senior researchers, in particular, are expected to take a lead in ensuring that research is fruitful and that results are either exploited commercially or made accessible to the public (or both) whenever the opportunity arises.\end{minipage}}
%
%}
%\TODO{Think about public engagement. I'm planning to set up a ``Data Science Network'' around Lancaster to help generate both enthusiasm and contacts within local companies.  Now might be a good time to write something more formally about it!}

With the launch of the Data Science Institute, Lancaster University will be inaugurating a ``Data Science Network'', in conjunction with Lancaster University's Knowledge Business Centre, an innovation hub providing a gateway for business/academic interaction which allows the transfer of expertise between Lancaster's academics, regional businesses and community partnerships through training and technology transfer activities.  This network will bring together academic data scientists with local companies in regular show and tell sessions.  Dr Gabillon will be a regular participant at these events, enabling bi-directional communication of opportunities and requirements, and the building of a network of industry contacts.
In addition, Lancaster University supports researchers to write for the Conversation, a news  service delivering articles directly from researchers to the public; Dr Gabilon will make use of this support to produce expository articles explaining the benefits that adaptive data science approaches can deliver to society.
Finally, to ensure successful public engagement, Dr Gabillon will attend Lancaster University's ``The Engaging Researcher Course'', a one-day experiential training course to explore public engagement activities that researchers can get involved in.


The excellent and innovative research generated in this project will of course be published Open Access in the world's leading academic journals and conferences, and all code generated will be also be released under standard Open frameworks.  Prof.\ Leslie currently works with several companies, both large and small, including the Defence Science and Technology Laboratory who have a current interest in security games. Dr Gabillon will be mentored to develop similar relationships.  He will also work with Security Lancaster (www.lancs.ac.uk/security-lancaster) to ensure the results of the current project are shared with relevant industrial and government partners.  We will discuss results directly with companies in Lancaster University's Knowledge Business Centre, an innovation hub providing a gateway for business/academic interaction which allows the transfer of expertise between Lancaster's academics, regional businesses and community partnerships through training and technology transfer activities.  A particularly successful mechanism deployed extensively at Lancaster is the industrially-sponsored MSc or PhD project, which allows the supervisor's research to be both developed and deployed directly within a company; Dr Gabillon will be encouraged to join appropriate supervisory teams to help both disseminate the project's research and develop an industrial research network to enhance his future career.  The Research Support Office of Lancaster University has extensive experience of industrial engagement and will assist in the management of IP and any patents that may arise from the research.

\section{Implementation}
\label{sec:implementation}

%\TODO{Show them: specific tasks and clearly-defined outputs/deliverables; host institution has capacity to support researcher; coherent workplan (including justification for the scheduling); metrics to assess progress; clear management structure (ie what is done beyond regular supervisor meetings); risk management and contingency plans; quality management procedures}

\subsection{Overall coherence and effectiveness of the work plan, including appropriateness of the allocation of tasks and resources}

%{\em
%Describe the different work packages. The proposal should be designed in such a way to achieve the desired impact. A Gantt Chart should be included in the text listing the following:
%\begin{itemize}
%\item Work Packages titles (for EF there should be at least 1 WP);
%\item List of major deliverables;\footnote{A deliverable is a distinct output of the action, meaningful in terms of the action?s overall objectives and may be a report, a document, a technical diagram, a software, etc.}\footnote{Deliverable numbers ordered according to delivery dates. Please use the numbering convention <WP number>.<number of deliverable within that WP>. For example, deliverable 4.2 would be the second deliverable from work package 4.}
%\item List of major milestones;\footnote{Milestones are control points in the action that help to chart progress. Milestones may correspond to the completion of a key deliverable, allowing the next phase of the work to begin. They may also be needed at intermediary points so that, if problems have arisen, corrective measures can be taken. A milestone may be a critical decision point in the action where, for example, the researcher must decide which of several technologies to adopt for further development.}
%\item Secondments if applicable.
%\end{itemize}
%The schedule should be in terms of number of months elapsed from the start of the project.
%}

\paragraph{Work packages}
\begin{description}
\item[WP1: Combinatorial bandits] Dr Gabillon will develop new approaches to combinatorial bandits (Objective 1), investigating the pure exploration problem and the online regret problem.  This WP builds upon current research of Dr Gabillon and can be completed in months 1--6.  This will result in {\bf Deliverable 1.1}, a paper on pure exploration in combinatorial bandits, and {\bf Deliverable 1.2}, a paper on regret in combinatorial bandits with submodular reward structures.
\item[WP2: Security games] Objectives 2 and 3 will be considered in this Work Package, which will develop algorithms for both pure exploration and online performance guarantees in security games.  This WP brings together the knowledge of the Researcher and the Supervisor, and will therefore also be started in Month 1.  However it will take longer to complete due to the greater level of novelty for Dr Gabillon, so will continue for 12 months.  {\bf Deliverable 2.1} is a paper on pure exploration in security games; {\bf Deliverable 2.2} is a paper on online regret bounds in security games.
\item[WP3: Combinatorial security games] Dr Gabillon will address Objective 4 with the development  of methods for security games with combinatorial action spaces. This package combines the results of WP1 and WP2.  After completion of WP1, Dr Gabillon will switch attention to working in combinatorial games in parallel with WP2, and this WP will continue until the end of the project.  {\bf Deliverable 3.1} is a paper presenting the results of this research.
\item[WP4: Network defence] Objective 5 will be addressed, with the application of combinatorial work (WP1 and WP3) to network problems. Dr Gabillon will visit Dr Michal Valko learn about techniques in networks and how to integrate them with the previously-developed combinatorial results.  He will also collaborate with researchers from Security Lancaster to develop applications in supply chain protection.  This package will be started after WP2 has been completed in month 12, and run until the end of the project. {\bf Deliverable 4.1} is a paper describing a bandit approach to network defence, and {\bf Deliverable 4.2} is a paper describing the game-theoretical results.
\end{description}

%\TODO{Other deliverable, eg industry engagement workshop, academic workshop?}

\paragraph{Major milestones}
\begin{description}
\item[Milestone 1] is the completion of WP1. If strong results are obtained in this first phase of research, then extending to combinatorial games under an assumption that the attacker always plays a best response to the current mixed strategy will be relatively straightforward, and greater emphasis can be placed on WP3 straight away.  If the results here are not so strong, more effort will need to be given to WP2 in order to obtain suitable building blocks for WP3.
\item[Milestone 2] is at the end of 1 year of the project, when WP1 and WP2 will both be completed, and WP3 is in progress. This will give an opportunity to take stock and decide the problems to be addressed in WP4. If the game-theoretical results in WP2 are strong, and WP3 is progressing well, then the full game-theoretical approach can be addressed directly in WP4.  However weaker game-theoretical results may necessitate initial focus on bandit approaches in WP4.
\end{description}

\subsection{Appropriateness of the management structure and procedures, including quality management and risk management}

%{\em
%Develop your proposal according to the following lines:
%\begin{itemize}
%\item Project organisation and management structure, including the financial management strategy, as well as the progress monitoring mechanisms put in place;
%\item Risks that might endanger reaching project objectives and the contingency plans to be put in place should risk occur.
%\end{itemize}
%}

The Research Support Office at Lancaster University has extensive experience of managing European project grants, and will be responsible for administering the project budget, legal aspects and potential commercial exploitation of the research.  Dr Gabillon will be a member of the Department of Mathematics and Statistics, and more specifically the Statistical Learning group lead by Prof.\ Leslie.  He will also be assigned a formal mentor under standard Lancaster University human resources procedures, who will be a second point of contact.  During the project, Dr Gabillon will be responsible for the research work, and will meet weekly with Prof.\ Leslie to discuss results, challenges and research strategies.  Dr Gabillon will also be responsible for the management of the project; he will be supervised in this task through monthly management and mentoring meetings with Prof.\ Leslie, in which progress against the workplan and career development plan will be discussed.

Clearly there are risks at each stage of an ambitious research project such as this. The two-pronged approach mitigates some of this risk: if developments in combinatorial approaches prove to be difficult then greater focus will be placed on the game theory, and vice versa.  That being said, both of WP1 and WP2 contain elements which are lower-risk while still likely to yield high-quality research outputs. A solid foundation can thus be laid while the Researcher and Supervisor develop a working relationship, in preparation for the more ambitious objectives in the latter part of the project.

\begin{figure}[htbp]
%{\em
%Gantt chart
%Reflecting work package, secondments, training events and dissemination / public engagement activities}
\begin{center}

\begin{ganttchart}[
    canvas/.append style={fill=none, draw=black!5, line width=.75pt},
    hgrid style/.style={draw=black!5, line width=.75pt},
    vgrid={*1{draw=black!5, line width=.75pt}},
    title/.style={draw=none, fill=none},
    title label font=\bfseries\footnotesize,
    title label node/.append style={below=7pt},
    include title in canvas=false,
    bar label font=\small\color{black!70},
    bar label node/.append style={left=2cm},
    bar/.append style={draw=none, fill=black!63},
    bar progress label font=\footnotesize\color{black!70},
    group left shift=0,
    group right shift=0,
    group height=.5,
    group peaks tip position=0,
    group label node/.append style={left=.6cm},
    group progress label font=\bfseries\small
  ]{1}{24}
  \gantttitle[
    title label node/.append style={below left=7pt and -3pt}
  ]{Month:\quad1}{1}
  \gantttitlelist{2,...,24}{1} \\
  \ganttgroup{WP1}{1}{6} \\
  \ganttgroup{Deliverable 1.1}{4}{4} \\
  \ganttgroup{Deliverable 1.2}{6}{6} \\
  \ganttgroup{WP2}{1}{12} \\
  \ganttgroup{Deliverable 2.1}{10}{10} \\
  \ganttgroup{Deliverable 2.2}{12}{12} \\
  \ganttgroup{Conference}{11}{11} \\
  \ganttgroup{WP3}{7}{24} \\
  \ganttgroup{Deliverable 3.1}{22}{22} \\
  \ganttgroup{WP4}{13}{24} \\
  \ganttgroup{Visit to Lille}{14}{14} \\
  \ganttgroup{Deliverable 4.1}{17}{17} \\
  \ganttgroup{Deliverable 2.2}{24}{24} \\
  \ganttgroup{Conference}{16}{16}\\
  \ganttgroup{Conference}{23}{23} \\
  \ganttgroup{Public engagement}{1}{24} \\
  \ganttgroup{Industry engagement}{1}{24}\\
  \ganttgroup{Milestone 1}{7}{7}\\
  \ganttgroup{Milestone 2}{13}{13}
\end{ganttchart}

\end{center}
\end{figure}

\subsection{Appropriateness of the institutional environment (infrastructure)}
\label{sec:institution}

%{\em
%\begin{itemize}
%\item Give a description of the main tasks and commitments of the beneficiary and partners (if applicable).
%\item Describe the infrastructure, logistics, facilities offered in as far they are necessary for the good implementation of the action.
%\end{itemize}
%}

Dr Gabillon will be hosted in the Department of Mathematics and Statistics, Lancaster University.  Prof.\ Leslie will provide the main mentorship and research supervision.  The Statistical Learning group, and the Statistics Research Group beyond that, will provide further immediate support to Dr Gabillon.  The Department has extremely strong links with research groups in Operations Research in Lancaster University Management School, through the STOR-i Centre for Doctoral Training, and with Computer Science, through the Data Science Institute.  Therefore multiple researchers in cognate areas will contribute to the project with informal mentorship and research leadership, as well as providing an environment with multiple relevant research seminars.  In terms of physical resources, the Department will provide high quality office space and standard IT facilities, including high performance computing, to allow Dr Gabillon to carry out the project.


\subsection{Competences, experience and complementarity of the participating organisations and institutional commitment}
\label{sec:competences}


%{\em
%The active contribution of the beneficiary to the research and training activities should be described. For GF also the role of partner organisations in Third Countries for the outgoing phase should appear. Additionally a letter of commitment shall also be provided in Section 7 (included within the PDF file of part B, but outside the page limit) for the partner organisations in Third Countries.
%NB: Each participant is described in Section 5. This specific information should not be repeated here.
%}

The Department of Mathematics and Statistics at Lancaster University was ranked fifth equal in the United Kingdom in the most recent Research Excellence Framework assessment.  The Department has a thriving research environment, with 50 faculty, 11 post-doctoral fellows, and 72 PhD students.  The Department has numerous government- and industry-funded research projects, many of which relate to industrially-motivated statistics and operations research and are related to the currently-proposed project.  The skill set of Dr Gabillon complements that of the Beneficiary by providing expertise in current algorithmic approaches to bandit algorithms and reinforcement learning.  The host institution in return provides expertise in statistical methodology appropriate to online inference, and game theoretical learning, and a strong track-record of working with industry to ensure the fundamental research is relevant and generates impact.  In addition Dr Gabillon will develop links with Security Lancaster (www.lancaster.ac.uk/security-lancaster/), in which researchers are currently addressing the security of supply chains using game-theoretical approaches, to both develop test cases for the current research project and build links with their network of industry and government collaborators.

\newpage
\section{CV of the Experienced Researcher}
\label{sec:cv}


During the course of my studies several invaluable experiences have greatly contributed to my desire to pursue a research-based career in Computer Science. I have had the opportunity to participate in stimulating research projects, in such areas as Machine Learning or Signal Processing.
% As a result, I have obtained a diverse research background which I seek to put in practice through a challenging and thus interesting post-doctoral position at  Learning Agents Research Group. 
From my early years as an undergraduate student I have tried to keep the balance between theory and application. After three years of intensive Mathematics and Physics studies I entered TELECOM SudParis, a Telecommunication engineering school. There, on the one hand my engineering education made me comfortable with programming (C/C++, Java) and Network issues (LANs, WANs) and on the other and I personally got involved in a research project on PCA algorithms which has lead to a publication at ICASSP 2009. In 2008, I continued with my graduate studies in Applied Mathematics as a master student with focus on Statistical Learning where I developed solid background Machine Learning theory (including a course on Graphical Models by Francis Bach and one on Reinforcement Learning by Rémi Munos). Still I completed my master with an internship at INRIA research lab where I applied statistical learning techniques to help design a realistic automatic ad-server for Orange Inc affiliated websites. This work has launched a collaboration which is still in progress.
 
My current research involves the investigation of machine learning techniques to create algorithms that, in some way, adapts to its users, or more generally learns from its environment. The approach is both theoretical and application oriented. A major objective in our algorithms development is to ensure our algorithms capture the real complexity of a problem and testing in practice their performances in real world problems. During my PhD, I investigated Reinforcement Learning (RL) which is a field where one tries to solve complex systems where an agent has to learn from its environment. More precisely, the focus was on a class of algorithms called ``Classification-based Policy Iteration'' (CBPI) which are algorithms that learn directly the policies as output of a classifier. Thus they avoid, as in the standard RL techniques, to define a policy through an associated value function as this value function is often poorly approximated. Therefore, this class of algorithms is expected to perform better than its value-based counterparts whenever the policies are easier to represent than their value functions. However, CBPI algorithms can require large number of samples from the environment. To improve the CBPI efficiency, I proposed new hybrid approaches using value function approximations in the CBPI framework that leverage the benefits of both approaches (which led to two publications in ICML 2011 \& 2012 while a journal paper has been published in JMLR). Moreover, we applied our techniques in the game of Tetris, a domain where RL techniques had obtained poor results, and learned a controller removing on average 50.000.000 lines (the best in the literature, to the best of our knowledge which is reported in a paper in NIPS 2013).

I also investigated Bandit problems. Bandit problems are core problems to model any problem involving adaptiveness. We designed a sampling strategy to solve several bandit problems in parallel (which led to two publications in NIPS 2011 \& 2012).

During the course of my Ph.D. I worked as an research intern for 6 months at Technicolor Labs in Palo Alto California under the supervision of Branislav Kveton.  Our primary goal was to improve the questionnaire asked to elicit movie preferences of users for a recommendation website. The problem was cast as an adaptive submodular maximization problem. The novelty was that we consider this problem in the case where the preferences of the users are not supposed to be known to build the questionnaire but need to be learned (which led to a publication in NIPS 2013).

As a post-doctorate in the Queensland University of Technology, under the supervision of Peter Bartlett, I am conducting research in online learning.  My first project deals with a combinatorial set of possible choices, is set in a stochastic setting and could model network routing problem (online shortest-path problem). The second one is set in the non-stochastic setting (adversarial) where the goal is to give a simple setting of this bandit game that admits an exact mimimax solution. This therefore is a more theoretical question that draws connection with game theory.

Through the experiences already described I developed my ability to work in a team environment. The international conferences, internships and summer schools I have been attending gave me the opportunity to learn and exchange with researchers from diverse horizons. In addition, teaching computer science (Algorithmic with Python \& Databases) for Master and Licence students keeps enriching my communication skills. I build up my programming skills through my curriculum in a telecommunication engineering school and later through the lectures and practical sections I gave. Moreover most of my projects have involved programming part which have made me comfortable with coding in Python and  C++.

My long term career goal is to become a  researcher.  I wish to gain professional experience at an environment that will allow me to expand my knowledge and capabilities through collaborations with researchers who can mentor and inspire me. I am confident that GRASP will provide me with such an environment and much more. It fits my willingness to  lead research that can find real applications, particularly in artificial intelligence for games or robotic purpose (as I already worked on the Tetris game). I believe that my background in Machine Learning will permit me to take on GRASP challenge on designing powerful lifelong learning algorithms. I also believe that my diverse research background, and my prior exposure to similar research environments make me a unique candidate for the internship program at GRASP. I look forward to conducting research at GRASP  world-class research environment, while nurturing my innovative and practical abilities.
 \begin{center} \textbf{Curriculum Vitae of the Applicant, Dr Victor Gabillon}  \end{center}
 
\noindent\textbf{Education}\\[-.4cm]\noindent\makebox[\linewidth]{\rule{\columnwidth}{0.4pt}}
\begin{vitem}{June 2014 }{PhD in Computer Science}
 in Team SequeL, INRIA Lille - Nord Europe, France.
%Defended on June 12, 2014\\
\textit{Title:} ``Budgeted Classification-based Policy Iteration''\\
Domains: Reinforcement learning \& Bandits games\\
Supervisors: Mohammad Ghavamzadeh \& Philippe Preux

 \end{vitem}


 \begin{vitem}{2008-09}{M.Sc. in applied mathematics, École Normale Supérieure, Cachan, France.}
Cursus MVA (image processing \& statistical learning) with honours.\\
Relevant courses: Reinforcement Learning, Graphical Models, Statistical learning (SVM, Boosting...).
 \end{vitem}

 \begin{vitem}{2006-08}{Engineering degree, TELECOM SudParis, Évry, France.}

Graduate school of engineering committed to the development of information technology. \\
Relevant courses: Programming, Statistics, Information Theory, Image Processing, LANs \& WANs.
 \end{vitem}
 
\subsection{Publications}
 Victor Gabillon, Branislav Kveton, Zheng Wen, Brian Eriksson $\&$ S. Muthukrishnan, \textbf{\emph{Large Scale Optimistic Adaptive Submodularity}}.
AAAI $2014$, $28^{th}$ Conference on Artificial Intelligence.
Oral presentation at Quebec City, Canada, July $2014$.


 Victor Gabillon, Mohammad Ghavamzadeh $\&$ Bruno Scherrer, 
\textbf{\emph{Approximate Dynamic Programming Finally Performs Well in the Game of Tetris}}.
NIPS $2013$, $27^{th}$ Conference on Neural Information Processing Systems.
Poster presentation at South Lake Tahoe, Nevada, December $2013$.


 Victor Gabillon, Branislav Kveton, Zheng Wen, Brian Eriksson $\&$ S. Muthukrishnan, \textbf{\emph{Adaptive Submodular Maximization in Bandit Setting}}.
NIPS $2013$, $27^{th}$ Conference on Neural Information Processing Systems.
Poster presentation at South Lake Tahoe, Nevada, December $2013$.


 Victor Gabillon, Mohammad Ghavamzadeh $\&$  Alessandro Lazaric, \textbf{\emph{Best Arm Identification: A unified approch to fixed budget and fixed confidence}}.
NIPS $2012$, $26^{th}$ Conference on Neural Information Processing Systems.
Poster presentation at South Lake Tahoe, Nevada, December $2012$.


 Bruno Scherrer, Mohammad Ghavamzadeh, Victor Gabillon $\&$ Matthieu Geist, \textbf{\emph{Approximate Modified Policy Iteration}}.
ICML $2012$, $29^{th}$  International Conference on Machine Learning.
Long lecture presentation at Edinburgh, Scotland, June $2012$.


 Victor Gabillon, Mohammad Ghavamzadeh, Alessandro Lazaric $\&$ Sébastien Bubeck, \textbf{\emph{Multi-Bandit Best Arm Identification}}.
NIPS $2011$, $25^{th}$ Conference on Neural Information Processing Systems.
Poster presentation at Granada, Spain, December $2011$.


 Victor Gabillon, Alessandro Lazaric, Mohammad Ghavamzadeh $\&$  Bruno Scherrer, \textbf{ \emph{Classification-based Policy Iteration with a Critic}}. ICML $2011$, $28^{th}$  International Conference on Machine Learning. Lecture presentation at Bellevue, USA, June $2011$.


 Victor Gabillon,  Alessandro Lazaric, Mohammad Ghavamzadeh \textbf{ \emph{Rollout Allocation Strategies for Classification-based Policy Iteration}}. Workshop on Reinforcement Learning and Search in Very Large Spaces International Conference on Machine Learning,  Lecture presentation at Haifa, Israel, June $2010$.


 Victor Gabillon, Jérémie Mary $\&$ Philippe Preux, \textbf{ \emph{Affichage de publicités sur des portails web}}. EGC $2010$, $10^{th}$ French-speaking International Conference on Knowledge Extraction and Management. Lecture presentation of long article at Hammamet, Tunisia, January $2010$. Best applied paper award.

 
 Jean-Pierre Delmas $\&$ Victor Gabillon,\textbf{ \emph{Asymptotic performance analysis of PCA algorithms based on the weighted subspace criterion}}.  ICASSP $2009$, International Conference on Acoustics, Speech and Signal Processing. Poster presentation at Taipei, Taiwan, April $2009$. 
   

  Bruno Scherrer, Mohammad Ghavamzadeh, Victor Gabillon $\&$ Matthieu Geist, \textbf{\emph{Approximate Modified Policy Iteration}}, JMLR.


\newpage
\section{Capacities of the Participating Organisations}
\label{sec:capacities}

%All organisations (whether beneficiary or partner organisation) must complete the appropriate table below, which will give input on the profile of the organisation as a whole. Complete one table of maximum one page for the beneficiary and half a page per partner organisation (min font size: 9). The experts will be instructed to disregard content above this limit.
\vspace{\baselineskip}

{\fontsize{9bp}{1em}\selectfont % should be 9pt
\noindent\begin{tabular}{>{\raggedright}p{.25\textwidth}p{.7\textwidth}}
  \multicolumn{2}{l}{\textbf{Beneficiary: Lancaster University}} \\\midrule
\textbf{General Description} & Lancaster University is a top ten UK university.  The Department of Mathematics and Statistics, within the Faculty
of Science and Technology, hosts one of the largest and strongest statistics research groups in the
UK comprising 25 academic staff, 10 research associates and around 50 FTE research students. In the 2014
Research Excellence Framework assessment, the Mathematical Sciences at Lancaster were ranked fifth overall and third in
terms of the impact of research.  Research is supported by grants from the UK Research Councils, the European Commission, and industrial sponsors. The statistics research group is also a fundamental partner in Lancaster's new Data Science Institute, which aims to act as a catalyst for Data Science, providing an end-to-end interdisciplinary research capability. 

\\\midrule
\textbf{Role and Commitment of key persons (supervisor)} &
Prof.\ David Leslie, PhD in Mathematics (University of Bristol, 2003).  17 PhD students and 2 post-doctoral fellows supervised. 5\% FTE time commitment to the project throughout the 24 month duration.
\\\midrule
\textbf{Key Research Facilities, Infrastructure and Equipment} &
The Department of Mathematics and Statistics is housed in dedicated space at Lancaster University.  The Researcher will be provided with office space and basic equipment within the Department. Researchers in have access to the Department's own computer support (2.6FTE computer technicians) and computer cluster (nearly 500 computer cores, 800GB of memory). These computing facilities are supplemented by access to Lancaster University's High-End Computing cluster (1700 computer cores, 8TB of memory, 32TB of high performance filestore).

\\\midrule
\textbf{Independent research premises?} & Yes

\\\midrule
\textbf{Previous Involvement in Research and Training Programmes} &
Between 2001 and 2005 the department held the Marie Curie Training Site status for its PhD programme. The Postgraduate Statistics Center (PSC) was founded in 2005 as the only Centre for Excellence in Teaching and Learning focussing on postgraduate statistics in the UK. The PSC is still operative and runs three Masters degrees (Statistics, Quantitative Methods, and Quantitative Finance) and coordinates the PhD programme in statistics.


\\\midrule
\textbf{Current involvement in Research and Training Programmes} &
Together with the Management School, the Department hosts and runs STOR-i, a
multi-million pound EPSRC-funded Centre for Doctoral Training in
Statistics and Operational Research in partnership with industry.  The
Centre was established in 2010 and funds 12 PhD students per year.  The department is also a key player in the Academy for Phd Training in Statistics, a collaboration between major UK statistics research groups to organise courses for first-year PhD students in statistics and applied probability nationally.  The group hosts one node of a multi-institution Programme Grant on Intractable Likelihood, and received industrial funding from companies including Shell, BT, Google and Unilever. The
Department's Medical and Pharmaceutical Statistics Research Unit
works closely with the pharmaceutical
industry and public sector research institutes to develop novel
statistical methods for the design and analysis of clinical trials. It
leads the EU-funded research training network IDEAS (www.ideas-itn.eu) and
is an integral part of the Medical Research Council funded North-West Hub for Trials
Methodology Research.
\\\midrule
\textbf{Relevant Publications and/or research/innovation products} &

 Perkins, S. and Leslie, D.S. (2014)  Stochastic fictitious play with continuous action sets. {\em Journal of Economic Theory} {\bf 152}, 179--213.

Chapman, A.C., Leslie, D.S., Rogers, A. and Jennings, N.R. (2013) Convergent learning algorithms for unknown reward games. {\em SIAM Journal on Control and Optimization} {\bf 51}, 3154-3180.

% Chapman, A.C., Leslie, D.S., Rogers, A. and Jennings, N.R. (2013) Learning in unknown reward games: Application to sensor networks.  {\em The Computer Journal} bxt082.

% Perkins, S. and Leslie, D.S. (2012) Asynchronous stochastic approximation with differential inclusions.  {\em Stochastic Systems} {\bf 2}, 409--446.
May, B.C., Korda, N., Lee, A. and Leslie, D.S. (2012) Optimistic Bayesian sampling in contextual-bandit problems. {\em Journal of Machine Learning Research} {\bf 13}, 2069--2106.

% Chapman, A.C., Rogers, A.C., Jennings, N.R. and Leslie, D.S. (2011)  A unifying framework for iterative approximate best response algorithms for distributed constraint optimisation problems.  {\em The Knowledge Engineering Review} {\bf 26}, 411--444.

Larsen, T., Leslie, D.S., Collins, E.J. and Bogacz, R. (2010) Posterior weighted reinforcement learning with state uncertainty.  {\em Neural Computation} {\bf 22}, 1149--1179.

% Rezek, I., Leslie, D.S., Reece, S., Roberts, S.J., Rogers, A., Dash, R.K., and Jennings, N.R. (2008) On similarities between inference in game theory and machine learning. {\em Journal of Artificial Intelligence Research} {\bf 33}, 259--283.

% {Leslie, D.S.} and Collins, E.J. (2006)  Generalised weakened fictitious    play.  {\em Games and Economic Behavior} {\bf 56}, 285--298.
% {Leslie, D.S.} and Collins, E.J. (2005)  Individual Q-learning in normal form games. {\em SIAM Journal on Control and Optimization} {\bf 44},    495--514.

{Leslie, D.S.} and Collins, E.J. (2003) Convergent multiple-timescales    reinforcement learning algorithms in normal form games. {\em Annals of    Applied Probability} {\bf 13}, 1231--1251.

\\\bottomrule
\end{tabular}}
\vspace{\baselineskip}

%{\fontsize{9bp}{1em}\selectfont
%\noindent\begin{tabular}{>{\raggedright}p{.25\textwidth}p{.7\textwidth}}
%  \multicolumn{2}{l}{\textbf{Partner Organisation Y}} \\\midrule
%\textbf{General Description} &
%
%\\\midrule
%\textbf{Key Persons and Expertise (supervisor)} &
%
%\\\midrule
%\textbf{Key Research facilities, infrastructure and equipment} &
%
%\\\midrule
%\textbf{Previous and Current Involvement in Research and Training Programmes} &
%
%\\\midrule
%\textbf{Relevant Publications and/or research/innovation product} &
%(Max 3)
%\\\bottomrule
%\end{tabular}}


\newpage
\vspace{15mm}
\begin{center}


        \Large{
      
     
        \textbf{ENDPAGE}
  
          \vspace{15mm}
          MARIE SKLODOWSKA-CURIE ACTIONS\\
          \vspace{1cm}
          
          \textbf{Individual Fellowships (IF)}\\
          \textbf{Call: H2020-MSCA-IF-2014}
          \vspace{2cm}                   

          PART B
          \vspace{2.5cm}

          ``\acronym''
          \vspace{2cm}

          \textbf{This proposal is to be evaluated as:}
          \vspace{.5cm}

          \textbf{[Standard EF]}
        }

  \end{center}
\vspace{1cm}


\end{document}