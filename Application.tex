\documentclass[a4paper,11pt]{article}

\usepackage[T1]{fontenc}
\usepackage{lmodern}


\usepackage[utf8]{inputenc}
\usepackage[french,english]{babel}

\usepackage{eurosym}
\usepackage{lastpage}
\usepackage{xspace}
\usepackage[margin=16mm,includehead,includefoot]{geometry}
\usepackage{fancyhdr}
\usepackage{booktabs,lipsum}
\usepackage{graphicx}
\usepackage{multirow}
\usepackage{array}
\usepackage{tabularx}
\usepackage{xcolor}
\usepackage{csquotes}
\usepackage{pgfgantt}
\usepackage{titlesec}
\usepackage[style=verbose-ibid,backend=bibtex]{biblatex}
\usepackage{hyperref}
\usepackage{amsmath}
\usepackage{amsfonts}
\usepackage{amssymb}
\usepackage{vgResume}

\newcommand{\TODO}[1]{{\textcolor{red}{[\textbf{TODO:} #1]}}}
\newcommand{\acronym}{{\sc OSEGA}\xspace}
\titlespacing\section{0pt}{6pt plus 4pt minus 2pt}{0pt plus 2pt minus 2pt}
\titlespacing\subsection{0pt}{6pt plus 4pt minus 2pt}{0pt plus 2pt minus 2pt}
\titlespacing\subsubsection{0pt}{6pt plus 4pt minus 2pt}{0pt plus 2pt minus 2pt}

\titleformat*{\section}{\large\bfseries}
\titleformat*{\subsection}{\normalsize\bfseries}
\titleformat*{\subsubsection}{\normalsize\bfseries}
\let\oldfootnotesize\footnotesize
\renewcommand{\footnotesize}{\fontsize{8bp}{1em}\selectfont}
\renewcommand{\cite}{\autocite} % citations in footnotes
\bibliography{biblio}

\headheight=14pt

\hypersetup{
    pdftitle={H2020-MSCA-IF-2014},    % title
    pdfauthor={Victor Gabillon},
    colorlinks=true,
    citecolor=black,
    linkcolor=black,
    urlcolor=blue
  }

\pagestyle{fancy}
\fancyhead{}
\fancyhead[C]{\acronym---Standard EF}
\fancyfoot{}
\fancyfoot[C]{Part B - Page \thepage~of \pageref{LastPage}}


\definecolor{resRouge}{RGB}{3,35,48}
\definecolor{resCas}{RGB}{255,54,35}
\definecolor{resNoir1}{RGB}{36,75,107}
\definecolor{resNoir2}{RGB}{255,54,35}
\definecolor{resNoir3}{RGB}{3,35,48}




\renewcommand{\headrulewidth}{0pt}


\renewcommand{\contentsname}{TABLE OF CONTENTS}


\newdimen\pHeight
\pHeight=-32678sp
\newdimen\pLower
\pLower=-4096sp
\newdimen\pLineWidth
\pLineWidth=32678sp
\newdimen\pKern
\pKern=-276480sp
\newdimen\pIR
\pIR=-131072sp
\newsavebox{\Cbox}
\newcommand{\HRule}{\rule{\linewidth}{0.5mm}}

\newsavebox{\vertCmplx}
\newdimen\Cheight
\newdimen\Cwidth
\sbox{\Cbox}{\rm C}
\Cheight=\ht\Cbox
\Cwidth=\wd\Cbox
\advance\Cheight by \pHeight
\sbox{\vertCmplx}{\rule[\pLower]{\pLineWidth}{\Cheight}}
\sbox{\Cbox}{\usebox{\Cbox}\kern\pKern\usebox{\vertCmplx}}
\wd\Cbox=\Cwidth
%\def\C{\usebox{\Cbox}}
%\def\Re{{\rm I\kern\pIR R}}
\def\Re{\mathbb{R}}
\def\Nat{{\rm I\kern\pIR N}}
%\def\argmax{\mathop{\rm arg\,max}}
%\def\argmin{\mathop{\rm arg\,min}}
\def\liminf{\mathop{\rm lim\,inf}}
\def\log{\mathop{\rm log}}
\def\Mean{{\bf Mean}}
\def\Var{{\bf Var}}
\def\Cov{{\bf Cov}}
\def\exptE{{\E}}
%\def\diag{\mathop{\rm diag}}

\newcommand{\bsmall}[1]{\vspace{#1}\begin{small}}
\newcommand{\esmall}[1]{\end{small}\vspace{#1}}



\newcommand{\argmax}{\operatorname*{argmax}} %\operatorname* pour les op. pouvant admettre des limites...
\newcommand{\argmin}{\operatorname*{argmin}}

\newcommand{\diag}{\operatorname*{diag}}

\newcommand{\greedy}{\operatorname*{{\cal G}}}
\newcommand{\lip}{\operatorname*{Lip}}
\renewcommand{\mod}{\operatorname*{mod}}

\newcommand{\bruno}[1]{~\\
{\bf[Comment (Bruno)]: \emph{#1}}
~\\
}

\newcommand\pp[1]{\Gamma^{#1}}

\def\defi{\stackrel{\Delta}{=}}
\def\E{\mathbb{E}}
\def\R{\mathbb{R}}
\def\G{{\cal G}}
\def\F{{\cal F}}
\def\bv{{\bf v}}
\def\bq{{\bf q}}
\def\bpi{\boldsymbol{\pi}}


\def\proj{{\cal A}}

\def\ie{{that is} }
\def\eg{{for instance} }
\def\cf{{see} }


\newcommand\mpi[1]{$\mbox{AMPI}_{#1}$}
\newcommand\cbmpi[1]{$\mbox{CBMPI}_{#1}$}



\def\A{{\mathcal{A}}}
\def\B{{\mathcal{B}}}
\def\C{{\mathcal{C}}}
\def\D{{\mathcal{D}}}
\def\E{{\mathbb{E}}}
\def\F{{\mathcal{F}}}
\def\G{{\mathcal{G}}}
%\def\H{{\mathcal{H}}}

\def\K{{\mathcal{K}}}
\def\I{{\mathcal{I}}}
\def\J{{\mathcal{J}}}
\def\L{{\mathcal{L}}}
\def\M{{\mathcal{M}}}
\def\N{{\mathcal{N}}}
\def\O{{\mathcal{O}}}
\def\P{{\mathcal{P}}}
\def\Q{{\mathcal{Q}}}
%\def\R{{\mathcal{R}}}
\def\S{{\mathcal{S}}}
\def\T{{\mathcal{T}}}
\def\U{{\mathcal{U}}}
\def\V{{\mathcal{V}}}
\def\W{{\mathcal{W}}}
\def\X{{\mathcal{X}}}
\def\Z{{\mathcal{Z}}}

\def\vecalpha{{\boldsymbol{\alpha}}}
\def\vecmu{{\boldsymbol{\mu}}}
\def\vectheta{{\boldsymbol{\theta}}}
\def\vecpi{{\boldsymbol{\pi}}}
\def\vecpsi{{\boldsymbol{\psi}}}
\def\vecphi{{\boldsymbol{\phi}}}
\def\vec0{{\boldsymbol{0}}}
\def\vecf{{\boldsymbol{f}}}
\def\veck{{\boldsymbol{k}}}
\def\vecp{{\boldsymbol{p}}}
\def\vecq{{\boldsymbol{q}}}
\def\vecu{{\boldsymbol{u}}}
\def\vecv{{\boldsymbol{v}}}
\def\vecw{{\boldsymbol{w}}}
\def\vecx{{\boldsymbol{x}}}
\def\vecy{{\boldsymbol{y}}}
\def\vecz{{\boldsymbol{z}}}

\def\matSigma{{\boldsymbol{\Sigma}}}
\def\matA{{\boldsymbol{A}}}
\def\matC{{\boldsymbol{C}}}
\def\matF{{\boldsymbol{F}}}
\def\matG{{\boldsymbol{G}}}
\def\matH{{\boldsymbol{H}}}
\def\matI{{\boldsymbol{I}}}
\def\matK{{\boldsymbol{K}}}
\def\matP{{\boldsymbol{P}}}
\def\matU{{\boldsymbol{U}}}
\def\matY{{\boldsymbol{Y}}}
\def\matZ{{\boldsymbol{Z}}}

\def\setP{{\mathbb{P}}}

\newcommand{\iid}{\stackrel{iid}{\sim}}

% Symbols
\newcommand{\alg}{\mathcal A}
\renewcommand{\Re}{\mathbb R}
%\def\argmax{\mathop{\rm arg\,max}}
%\def\argmin{\mathop{\rm arg\,min}}
\def\arginf{\mathop{\rm arg\,inf}}
% MDP notation
\newcommand{\MDP}{\mathcal M}
\newcommand{\discount}{\gamma}
\newcommand{\state}{\S}
\newcommand{\action}{\mathcal{A}}
\newcommand{\reward}{r}
\newcommand{\Return}{R}
%\newcommand{\truncReturn}{\overline{R}}
\newcommand{\truncReturn}{R}
\newcommand{\VFReturn}{\widetilde{R}}
\newcommand{\dynamics}{p}
%\newcommand{\reward}{\mathcal{R}}
%\newcommand{\dynamics}{\mathcal{P}}

\newcommand{\rolSetDistri}{\mu}

\newcommand{\RolloutSize}{m}

\newcommand{\pol}{\pi}
\newcommand{\polSpace}{\Pi}
\newcommand{\polDist}{\delta}
\newcommand{\fun}{f}
\newcommand{\Data}{\mathcal{D}}
\newcommand{\funSpace}{\mathcal{F}}
\newcommand{\vSpace}{\mathcal{B}^v}
\newcommand{\qSpace}{\mathcal{B}^Q}
\newcommand{\vPiSpace}{\mathcal{B}^{\pol}}
%\newcommand{\pseudoDim}{\mathcal{V}^+}
\newcommand{\pseudoDim}[1]{V_{#1^+}}
\newcommand{\cover}{\mathcal{N}}

\newcommand{\Rmax}{R_{\max}}

\newcommand{\Vfun}{v}
%\newcommand{\Vmax}{V_{\max}}
\newcommand{\Vmax}{V_{\max}}
\newcommand{\hV}{\widehat{\Vfun}}
\newcommand{\hv}{\widehat{\Vfun}}
\newcommand{\bV}{\overline{\Vfun}}

\newcommand{\Qfun}{Q}
\newcommand{\estQfun}{\widehat{\Qfun}}
\newcommand{\Qmax}{Q_{\max}}
%\newcommand{\Qmax}{q}
\newcommand{\hQ}{\widehat{Q}}

\newcommand{\hmu}{\widehat{\rho}}
\newcommand{\hrho}{\widehat{\mu}}
\newcommand{\hell}{\widehat{\ell}}

\newcommand{\hhmu}{\widehat{\mu}}

\newcommand{\greedyPol}{\mathcal G}
\newcommand{\data}{\mathcal D}
%\newcommand{\ibar}{\bar{i}}
\newcommand{\ibar}{j}

\newcommand{\ind}[1]{\mathbb I\left\lbrace {#1} \right\rbrace}

\newcommand{\norm}[1]{\Arrowvert{#1}\Arrowvert}
\newcommand{\normMu}[1]{||{#1}||_{1,\mu}}
\newcommand{\normNu}[1]{||{#1}||_{1,\nu}}
\newcommand{\err}{\varepsilon}
%\newcommand{\expErr}{\ell}
%\newcommand{\empErr}{\widehat{\ell}}
\newcommand{\expErr}{\epsilon'}
\newcommand{\empErr}{\widehat{\epsilon}'}

\newcommand{\lateDist}{{\rho}}
\newcommand{\initDist}{{\mu}}
\newcommand{\nSamplesDPI}{{N'}}
\newcommand{\sumSamplesDPI}{\sum_{i=1}^{\nSamplesDPI}}
\newcommand{\nSamples}{{N'}}
\newcommand{\sumSamples}{\sum_{i=1}^{\nSamples}}
\newcommand{\nRolls}{M}
\newcommand{\sumRolls}{\sum_{j=1}^{\nRolls}}
\newcommand{\avgActions}{\frac{1}{|\action|}\sum_{a\in\action}}
\newcommand{\sumActions}{\sum_{a\in\action}}

\newcommand{\probParam}{\delta}
% Indecies, number of elements, summations, products, and ranges
\newcommand{\idxTask}{m}
\newcommand{\idxTaskB}{m'}
\newcommand{\idxTaskNew}{\nTasks+1}
\newcommand{\idxSample}{n}
\newcommand{\idxST}{mn}
\newcommand{\idxClass}{c}
\newcommand{\nTasks}{M}
\newcommand{\nTasksClass}{\nTasks_{\class}}
\newcommand{\nTasksClassBut}{\nTasks_{-\idxTask,\class}}
\newcommand{\nReplica}{{N_{r}}}
\newcommand{\nClasses}{C}
\newcommand{\sumTask}{\sum_{\idxTask=1}^{\nTasks}}
\newcommand{\sumSample}{\sum_{\idxSample=1}^{\nSamples}}
\newcommand{\sumClass}{\sum_{\idxClass=1}^{\nClasses}}
\newcommand{\prodTask}{\prod_{\idxTask=1}^{\nTasks}}
\newcommand{\prodSample}{\prod_{\idxSample=1}^{\nSamples}}
\newcommand{\prodClass}{\prod_{\idxClass=1}^{\nClasses}}
\newcommand{\rangeTask}{\idxTask=1,\ldots,\nTasks}
\newcommand{\rangeSample}{\idxSample=1,\ldots,\nSamples}
\newcommand{\rangeClass}{\idxClass=1,\ldots,\nClasses}
\newcommand{\rangeClassInf}{\idxClass=1,\ldots,\infty}

\newcommand{\iter}{t}
\newcommand{\class}{c}
\newcommand{\classVectIter}{\textbf{c}^{(\iter)}}
\newcommand{\classVect}{\textbf{c}}
\newcommand{\classTask}{\class_{\idxTask}}
\newcommand{\classTaskB}{\class_{\idxTask'}}
\newcommand{\classTaskNew}{\class_{\idxTaskNew}}
\newcommand{\meanClass}{\mean_{\class}}
\newcommand{\precClass}{\precis_{\class}}
\newcommand{\covClass}{\cov_{\class}}
\newcommand{\varClass}{\var_{\class}}
\newcommand{\meanClassTask}{\mean_{\classTask}}
\newcommand{\precClassTask}{\precis_{\classTask}}
\newcommand{\covClassTask}{\cov_{\classTask}}
\newcommand{\varClassTask}{\var_{\classTask}}

\newcommand{\vfun}{V}
\newcommand{\vfunTask}{\vfun_{\idxTask}}
\newcommand{\vfunTMean}{\mean_{\vfunT}}
\newcommand{\vfunTCov}{\cov_{\vfunT}}
\newcommand{\vfunTMeanPost}{\meanPost_{\vfunT}}
\newcommand{\vfunTCovPost}{\covPost_{\vfunT}}

\newcommand{\xtest}{x_*}

\newcommand{\w}{{\bold w}}
\newcommand{\wAvg}{\bar{\w}}
\newcommand{\wFake}{\widehat{\w}}
\newcommand{\wTask}{\w_{\idxTask}}
\newcommand{\wTaskB}{\w_{\idxTask'}}
\newcommand{\wTaskNew}{\w_{\idxTaskNew}}
\newcommand{\wFakeTask}{\wFake_{\idxTask}}
\newcommand{\wFakeTaskNew}{\wFake_{\idxTaskNew}}
\newcommand{\wFakeTaskB}{\wFake_{\idxTask'}}
\newcommand{\wTaskMean}{\mean_{\idxTask}}
\newcommand{\wTaskCov}{\cov_{\idxTask}}
\newcommand{\wTaskPrec}{\precis_{\idxTask}}
\newcommand{\wTaskMeanPost}{\meanPost_{\idxTask}}
\newcommand{\wTaskPrecPost}{\precPost_{\idxTask}}
\newcommand{\wTaskCovPost}{\covPost_{\idxTask}}


\newcommand{\latent}{\mathcal Z}

\newcommand{\coeff}{\alpha}
\newcommand{\coeffT}{\coeff_{\task}}
\newcommand{\coeffTMean}{\mean_{\coeff}}
\newcommand{\coeffTCov}{\cov_{\coeff}}
\newcommand{\coeffTMeanPost}{\meanPost_{\coeffT}}
\newcommand{\coeffTCovPost}{\covPost_{\coeffT}}

\newcommand{\nfeatures}{s}
\newcommand{\dims}{d}

\newcommand{\featFun}{\phi}
\newcommand{\featVec}{\boldsymbol{\phi}}
\newcommand{\featMat}{\boldsymbol{\Phi}}
\newcommand{\featMatTask}{\featMat_{\idxTask}}
\newcommand{\featMatST}{\featMat_{\idxST}}

\newcommand{\kernelFun}{k}
\newcommand{\kernelFunT}{k_{\task}}
\newcommand{\kernelMat}{K}
\newcommand{\kernelMatT}{\kernelMat_{\task}}

\newcommand{\inSpace}{\mathcal X}
\newcommand{\inElemTask}{X_{\idxTask}}
\newcommand{\inElem}{x}
\newcommand{\inElemST}{\inElem_{\idxST}}
\newcommand{\inElemSTB}{\inElem'_{\idxST}}

\newcommand{\obs}{r}
\newcommand{\obsTask}{R_{\idxTask}}
\newcommand{\obsTaskB}{R_{\idxTask'}}
\newcommand{\obsTaskNew}{R_{\idxTaskNew}}

\newcommand{\outT}{R_{\idxTask}}
\newcommand{\outElemT}{r^i_\idxTask}
\newcommand{\outElem}[2]{r_{#2}^{#1}}

\newcommand{\noise}{N}
\newcommand{\noiseT}{\noise_{\task}}
\newcommand{\noiseTVar}{\Sigma_{\task}}
\newcommand{\noiseTVarB}{\var\transfT\transpose{\transfT}}
\newcommand{\stddev}{\sigma}

% Partial equations
\newcommand{\featHTTask}{\featMatTask\transpose{\matH}}
\newcommand{\HfeatTTask}{\matH\transpose{\featMat}_{\idxTask}}
\newcommand{\HfeatTTaskSample}{\matH\transpose{\featMat}_{\idxST}}
\newcommand{\invHHT}{\invA{{\matH\transpose{\matH}}}}
\newcommand{\HHT}{\matH\transpose{\matH}}

% Distributions and parameters
\newcommand{\normal}{\mathcal N}
\newcommand{\mean}{\boldsymbol{\mu}}
\newcommand{\meanPrior}{\mean_0}
\newcommand{\meanPost}{{\meanPrior'}}
\newcommand{\meanPriorRep}{\widetilde{\mean}_0}
\newcommand{\meanPostRep}{{\meanPriorRep'}}
\newcommand{\meanClassPrior}{\mean_{0\idxClass}}
\newcommand{\meanClassPriorRep}{\widetilde{\mean}_{0\idxClass}}
\newcommand{\meanClassPostRep}{{\meanClassPriorRep'}}

\newcommand{\var}{\sigma^2}
\newcommand{\varPrior}{\var_0}
\newcommand{\varPost}{\varPrior'}
\newcommand{\varPriorRep}{\widetilde{\var}_0}
\newcommand{\varPostRep}{\varPriorRep'}
\newcommand{\varClassPrior}{\var_{0\idxClass}}
\newcommand{\varClassPriorRep}{\widetilde{\var}_{0\idxClass}}
\newcommand{\varClassPostRep}{\varClassPriorRep'}

\newcommand{\cov}{\mathbf{\Sigma}}
\newcommand{\covB}{\mathbf{\Sigma}'}
\newcommand{\covPrior}{\cov_0}
\newcommand{\covPost}{\covPrior'}
\newcommand{\covPriorRep}{\widetilde{\cov}_0}
\newcommand{\covPostRep}{\covPriorRep'}
\newcommand{\covClassPrior}{\cov_{0\idxClass}}
\newcommand{\covClassPriorRep}{\widetilde{\cov}_{0\idxClass}}
\newcommand{\covClassPostRep}{\covClassPriorRep'}

\newcommand{\precis}{\Lambda}
\newcommand{\precPrior}{\precis_0}
\newcommand{\precPost}{\precisPrior'}
\newcommand{\precPriorRep}{\widetilde{\precis}_0}
\newcommand{\precPostRep}{\precisPriorRep'}
\newcommand{\precClassPrior}{\precis_{0\idxClass}}
\newcommand{\precClassPriorRep}{\widetilde{\precis}_{0\idxClass}}
\newcommand{\precClassPostRep}{\precisClassPriorRep'}

\newcommand{\iwishart}{\mathcal {IW}}
\newcommand{\wishart}{\mathcal W}

\newcommand{\wishScale}{k}
\newcommand{\wishScalePrior}{\wishScale_0}
\newcommand{\wishScalePost}{\wishScalePrior'}
\newcommand{\wishScalePriorRep}{\widetilde{\wishScale}_0}
\newcommand{\wishScalePostRep}{\wishScalePriorRep'}
\newcommand{\wishScaleClassPrior}{\wishScale_{0\idxClass}}
\newcommand{\wishScaleClassPriorRep}{\widetilde{\wishScale}_{0\idxClass}}
\newcommand{\wishScaleClassPostRep}{\wishScaleClassPriorRep'}

\newcommand{\freedom}{\nu}
\newcommand{\freedomPrior}{\freedom_0}
\newcommand{\freedomPost}{\freedomPrior'}
\newcommand{\freedomPriorRep}{\widetilde{\freedom}_0}
\newcommand{\freedomPostRep}{\freedomPriorRep'}
\newcommand{\freedomClassPrior}{\freedom_{0\idxClass}}
\newcommand{\freedomClassPriorRep}{\widetilde{\freedom}_{0\idxClass}}
\newcommand{\freedomClassPostRep}{\freedomClassPriorRep'}

\newcommand{\covNoise}{\mathbf{S}}

\newcommand{\igamma}{\mathcal {IG}}
\newcommand{\igScale}{\beta}
\newcommand{\igScalePrior}{\igScale_0}
\newcommand{\igScalePost}{\igScalePrior'}
\newcommand{\igScalePriorRep}{\widetilde{\igScale}_0}
\newcommand{\igScalePostRep}{\igScalePriorRep'}
\newcommand{\igScaleClassPrior}{\igScale_{0\idxClass}}
\newcommand{\igScaleClassPriorRep}{\widetilde{\igScale}_{0\idxClass}}
\newcommand{\igScaleClassPostRep}{\igScaleClassPriorRep'}

\newcommand{\igShape}{\alpha}
\newcommand{\igShapePrior}{\igShape_0}
\newcommand{\igShapePost}{\igShapePrior'}
\newcommand{\igShapePriorRep}{\widetilde{\igShape}_0}
\newcommand{\igShapePostRep}{\igShapePriorRep'}
\newcommand{\igShapeClassPrior}{\igShape_{0\idxClass}}
\newcommand{\igShapeClassPriorRep}{\widetilde{\igShape}_{0\idxClass}}
\newcommand{\igShapeClassPostRep}{\igShapeClassPriorRep'}

\newcommand{\mnomial}{\mathcal M}
\newcommand{\dirac}{\delta}

% Dirichlet process parameters and symbols
\newcommand{\baseDistro}{G_0}
\newcommand{\baseDistroPost}{\widetilde{G}}
\newcommand{\concentration}{\tau}
\newcommand{\stick}{v}
\newcommand{\stickClass}{\stick_\idxClass}
\newcommand{\mnCoeff}{\pi}
\newcommand{\mnCoeffClass}{\mnCoeff_\idxClass}

\newcommand{\param}{\theta}
\newcommand{\paramClass}{\param_\idxClass}
\newcommand{\paramClassTask}{\param_{\classTask}}
\newcommand{\hyper}{\psi}
\newcommand{\hyperPrior}{\hyper_0}
\newcommand{\hyperPriorRep}{\widetilde{\hyper}_0}
\newcommand{\hyperPost}{\hyperPrior'}
\newcommand{\hyperPostRep}{\hyperPriorRep}
\newcommand{\hyperClass}{\hyper_{0\idxClass}}
\newcommand{\hyperClassPost}{\hyperClass'}
\newcommand{\hyperClassRep}{\widetilde{\hyper}_{0\idxClass}}
\newcommand{\hyperClassPostRep}{\hyperClassRep'}

% Special matrices and operators

\newcommand{\matHT}{\transpose{\matH}}
\newcommand{\vectH}{h}
\newcommand{\identity}{I}
\newcommand{\identityA}[1]{I_{#1}}
\newcommand{\transpose}[1]{{#1}^\top}
\newcommand{\inv}[1]{#1^{-1}}
\newcommand{\invA}[1]{\left(#1\right)^{-1}}
\newcommand{\trace}[1]{\text{tr}\left(#1\right)}
\newcommand{\noisevec}{\mathcal E}

% Modified LQR parameters
\newcommand{\svar}{\eta^2}
\newcommand{\avar}{\delta^2}
\newcommand{\amean}{\lambda}

% Expectation, covariance and probability
\newcommand{\expectShort}{\mathbb E}
\newcommand{\expectA}[1]{\mathbb E \left[ {#1} \right]}
\newcommand{\expectB}[2]{\mathbb E_{#1} \left[ {#2} \right]}
\newcommand{\covA}[1]{\text{Cov} \left[ {#1} \right]}
\newcommand{\probA}[1]{\mathbb P\left[ {#1} \right]}
\newcommand{\probB}{p}

% Extras
\newcommand{\new}{\textrm{\scriptsize new}}
\newcommand{\paragTitle}[1]{\textbf{#1.}}
%\newcommand{\note}[1]{\textbf{Note.} \textit{#1}}
%\newcommand{\TODO}[1]{(\textbf{TODO: {#1}})}
\newcommand{\shrink}{\!\!\!}
%\newcommand{\eqref}[1]{(\ref{#1})}

%\newcommand{\errExpSTL}[1][]{er_{\task_{#1}}}
%\newcommand{\errExpSTLOpt}[1][]{er^*_{\task_{#1}}}
%\newcommand{\dotprod}[2]{\langle #1, #2 \rangle}

\newcommand{\LinSpace}{\F}
\newcommand{\TruncSpace}{\widetilde{\LinSpace}}
\newcommand{\distri}{\mu}
\newcommand{\empDistri}{\widehat{\distri}}

\newcommand{\q}{Q}

\newcommand{\appdx}[1]{Appendix~{#1}}
\newcommand{\eqrefb}[1]{Equation~\eqref{eq:#1}}


\def\AMPIV{\mbox{AMPI-V }}
\def\AMPIQ{\mbox{AMPI-Q }}


\def\szl{Szita and L{\H{o}}rincz }
\def\Ev{{\mathcal{E}}}



\begin{document}

\phantom{a}
\vspace{15mm}
\begin{center}


        \Large{
      
     
        \textbf{START PAGE}
  
          \vspace{15mm}
          MARIE SKLODOWSKA-CURIE ACTIONS\\
          \vspace{1cm}
          
          \textbf{Individual Fellowships (IF)}\\
          \textbf{Call: H2020-MSCA-IF-2015}
          \vspace{2cm}                   

          PART B
          \vspace{2.5cm}

          ``\acronym''
          \vspace{2cm}

          \textbf{This proposal is to be evaluated as:}
          \vspace{.5cm}

          \textbf{[Standard EF]}
        }

  \end{center}
\vspace{1cm}

\newpage
\setcounter{tocdepth}{1}
\setcounter{section}{-1}
\tableofcontents


\newpage
\section{List of Participants}
\label{sec:participants}

\newcommand\rotx[1]{\rotatebox[origin=c]{90}{\textbf{#1}}}
\newcommand\roty[1]{\rotatebox[origin=c]{90}{\parbox{4cm}{\raggedright\textbf{#1}}}}
\newcommand\MyHead[2]{\multicolumn{1}{l|}{\parbox{#1}{\centering #2}}}

\noindent\begin{tabular}{|m{2.4cm}|m{1cm}|b{1em}|b{1em}|c|m{2.5cm}|m{2cm}|c|}
\hline
  \textbf{Participants}
& \MyHead{1cm}{\textbf{Legal\\Entity\\Short\\Name}}
& \rotx{Academic}
& \rotx{Non-academic}
& \textbf{Country}
& \MyHead{2.1cm}{\textbf{Dept. / \\Division / \\Laboratory}}
& \textbf{Supervisor}
& \MyHead{2.5cm}{\textbf{Role of\\Partner\\Organisation}} \\
\hline
\underline{Beneficiary} & & & & & & & \\\hline
- NAME  & & & & & & & \\\hline
\underline{Partner} \underline{Organisation} & & & & & & & \\\hline
- NAME  & & & & & & & \\\hline
\end{tabular}
\vspace{\baselineskip}

Data for non-academic beneficiaries

\noindent\begin{tabular}{|m{1.7cm}|m{2cm}|m{1.8cm}|c|c|m{2.5cm}|c|c|c|}
\hline
  \textbf{Name}
& \roty{Location of research premises (city / country)}
& \roty{Type of R\&D activities}
& \roty{No. of fulltime employees}
& \roty{No. of employees in R\&D}
& \roty{Website}
& \roty{Annual turnover (approx. in Euro)}
& \roty{Enterprise status (Yes/No)}
& \roty{SME status  (Yes/No)}
\\\hline
& & & & & & & & \\\hline
\end{tabular}
\vspace{\baselineskip}

Note that:
\begin{itemize}
\item Any inter-relationship between different participating institutions or individuals (e.g. family ties, shared premises or facilities, joint ownership, financial interest, overlapping staff or directors, etc.) must be declared and justified in this part of the proposal;
\item The information in the table for non-academic beneficiaries must be based on current data, not projections;
\item The data provided relating to the capacity of the participating institutions will be subject to verification during the Grant Agreement preparation phase.
\end{itemize}


\newpage

\section{Excellence}
\label{sec:excellence}



\subsection{Quality, innovative aspects and credibility of the research} % (including inter/multidisciplinary aspects)}
\label{sec:quality}

A critical concern in the modern world is security. Effectively protecting ports, airports, trains and other transportation systems from malicious attacks, combating the trafficking of drugs, firearms and even people, and securing proprietary and sensitive information over the ever-growing cyber-networks, comprise some of the principal axes of this critical task. The main challenge in all of these problems is that maximum security must be obtained with a limited number of available resources. For instance, the total number of security agents is typically less than the number of targets that need to be protected. %available to simultaneously protect a multitude of designated targets may not be sufficient to provide full security coverage at an airport. 
This calls for the design of appropriate resource allocation techniques which optimise security under the constrained resources available, in the presence of uncertainty about the adversaries' interests. %Clearly, an important feature of these methods must be to provide highly unpredictable strategies, while devising appropriate target priorities in the presence of uncertainty about the adversaries' interests.  
 
Security resource allocation and scheduling problems comprise one of the many application areas that have recently been shown to greatly benefit from game-theoretic approaches. Indeed, as a solid mathematical framework to model strategic decision making, game theory has proved useful in many real-world applications from economics and political science to logic, computer science and psychology. In this paradigm, the problem is cast as a ``game'' and the objective is to find a solution whereby each ``player'' makes choices to maximise her own \textit{utilities}, which may often be in conflict with those of her opponent. A ``security game'' corresponds to a competition between a defender and an attacker. To solve a security game, all possible actions (attacks and defences) of the two players are enumerated, and for each player an outcome (value) is assigned, which depends on the pair of actions taken by both players. In cases where these outcomes are known, game-theoretic approaches have provided impressive results. Since 2007, the so-called ARMOR software \cite{pita2008deployed} has been used at Los Angeles International Airport to effectively determine checkpoints on roadways leading to the airport, and to determine canine patrol routes within terminals. Similar deployments have been made by the US Federal Air Marshals\cite{tsai2009iris}, the US coast guard\cite{shieh2012protect}, and to design the Los Angeles Metro system's fare inspection strategy\cite{yin2012trusts}. 

A severe limitation of these models is that they assume the utility functions to be known, whereas they must actually be estimated by experts or obtained from historical data. As a result, potentially high estimation errors or a lack of historical data (which is inevitable in a quickly-evolving security scenario) may render the security game solver useless. 
%However, the data from which these values can be obtained are usually scarce and the error in expert guesses may be high, rendering the security game solver useless. 
%A common example is when the players cannot completely foresee the outcome of some of her actions, or is bound to ignore what utilities her opponent is aiming to maximise. For instance, the exact efficacy of check-points or the drug smugglers' routes at an airport may be unknown. 
%Indeed, an important observation, forming the basis of this grant proposal, is that this {\em learning} problem lying at the heart of security games can be solved using carefully designed machine learning techniques. 
Therefore it is of importance to use methods that can quickly collect the most relevant data  in order to estimate the parameters of the game and quickly reach a satisfactory operational performance.

%Machine learning lies at the crossroad between statistics and computer science. The common goal is to design programs able to actively and intelligently gather data and extract information from the collected data, autonomously using them to make strategic decisions. Based on theoretically sound statistical methods, machine learning techniques are ubiquitously being deployed in a variety of modern applications ranging from robotics to personalised product recommendation. 

%\TODO{rewrite this end!!}

%The purpose of this proposal is to create \textit{practical}, \textit{scalable} and \textit{robust} methods for security games. First, we target \textit{practicality}  in the sense that our algorithms would be autonomous in handling the uncertainty in the model and would actively be working at reducing it by interacting with the environment in which the game takes place. Specifically, we aim to broaden the scope of repeated security game problems where the initial uncertainty about the players utilities can be overcome through using learning techniques  in conjunction with  repetitive plays of the game. 
%These techniques are from the extremely active field of machine learning research. Second, we target \textit{scalability} so that the solvers could handle a potentially  extremely large number of possible actions. To that end, simplifying structure assumptions such as combinatorial structure will be considered as well as the simplifying submodular property of the objective function.
%Finally the \textit{robustness}  is a key issue in security games. We propose to insure robustness in  three different ways. First, one can be  very conservative and assume that an adversary actually chooses (in the most adversarial way) the data that we do not know. Second, we introduce offline learning  where the test and the learning happen  before the strategies is actually use in the real world. Third, we might be required to learn defence strategies that do not possess large variances in their performance.

This project will therefore deploy approaches of {\em statistics} and {\em machine learning} within the framework of security games which are played repeatedly between a defender and attackers. Repeated security games allow for the continuous collection of data, which can in turn be used to estimate the parameters of the game and influence future behavour. %Another possibility is in fact that the defender can in turn test the game and collect (in the most efficient and less costly manner) more information about the game. 
Our key objective is to design efficient and theoretically sound, data-driven methods that can actively interact with the environment to {\em learn} and {\em act} in security games of realistic scales, which must therefore be \textit{practical}, \textit{scalable} and \textit{robust}. 

 

 

\subsubsection*{State-of-the-art}
\paragraph{Security meets Game Theory }
%\TODO{better distinction between minimax nash stacjkelberg}
From a game-theoretic perspective, a security problem is viewed as a two-player game that captures the interaction between a defender (e.g., border patrols, metro inspectors, network administrators) and attackers (e.g., terrorists/smugglers, illegal metro users, malicious cyber attackers). The action of the defender (attacker) is defined as selecting a subset of targets to protect (attack). For each defender/attacker action pair, \textit{utilities} are defined as the players' gain or loss, and the players' objectives are to maximise their corresponding pay-offs. From the defender's perspective, this corresponds to efficiently allocating a limited number of resources to secure some predefined targets. The expected utilities for both players can be stored in two  matrices,  $\boldsymbol A$ for the defender,  $\boldsymbol B$ for the attacker. Entries $\boldsymbol A_{i,j}$ and  $\boldsymbol  B_{i,j}$  are the expected utilities for the defender and attacker, respectively, when the defender plays strategy $i$ and the attacker responds with strategy $j$.
%The defender allocates a limited number of resources to secure some predefined target, which may under threat by the attacker. 
Solutions to such games rely on randomised (mixed) strategies, making each player's behaviour unpredictable to the other.  If a fully competitive setting, in which the attacker's gain is the defender's loss (i.e.\ the so-called zero-sum games) a fully robust strategy can be calculated for the defender, in that it provides guaranteed performance against {\em any} possible attacker, even if the defender's strategy is completely revealed to the attacker.  
A generalisation to more general (non zero-sum) games, which is particularly relevant to security games, is the Stackelberg equilibrium\cite{korzhyk2011stackelberg} in which the defender's mixed strategy is first publicised and the attacker plays a best response to this mixed strategy.
It is this Stackelberg security game framework that will be considered in the proposed project.
% Indeed, this is well-suited to the scenario where the defender's strategies may be at risk of revelation, leak or discovery through repetitive interactions. 
%Beyond these fundamental results, 
%some questions forming the primarouy focus of research in security games have been scalability, robustness with respect to uncertainty on the players utilities, or devising strategies that take advantage of the attacker's potentially limited rationality or bounded memory\cite{tambe2012game}. 

%\noindent \textbf{\textit{\\Tools from Sequential decision-making under uncertainty}}\\

\paragraph{Uncertainty in Security Games}
Most standard game-theoretical analyses assume that the payoff matrices are known in advance.  However uncertainty is endemic in most real-world applications. 
%Unlike standard game-theoretical approaches, in some scenarios, the players is uncertain about their utilities (the $A_{i,j}$'s), or those of the other players'.
For instance, in the context of security games, the random selection of passengers for security checks at an airport is a source of uncertainty in this game, where the outcome is random and the probability of successful security enforcement  is unknown. 
As also confirmed by several empirical studies in fraud and cybercrime detection, this phenomenon can significantly decrease the defender's performance\cite{granick2005faking, swire2009no}.
Extensive studies have been dedicated to the design of security games that are robust with respect to uncertainty about the environment\cite{aghassi2006robust,Nguyen14RO, Kiekintveld:2013}. 
However, an important observation is that much more can be done in the case of {\em repeated} security games.  
Indeed, this repetition allows the defender to further reduce her uncertainty about the model and intelligently {\em learn} how to improve her performance over time. Specifically, in this case, a security game solver can autonomously take intelligent decisions at repeated instances of the game, by carefully collecting, extracting and acting upon information from historical data. As discussed further in below, this is precisely the area where mathematical machine learning has the strongest results.

%\noindent \textbf{\textit{\\iii. Indispensable Tools from Machine Learning}}\\
%Machine learning is a field of computer science where the goal is to design software able to extract information from data so that the machine itself can make use of this information to take  autonomous decisions.
%The problem of sequential decision-making arises in our everyday lives, when we seek answers to such simple questions as how to navigate to work, retrieve our desired information from the Internet, or even
%play and win at backgammon, poker or tetris (the latter was used as a testbed in the author's PhD thesis). Naturally, these situations incur uncertainty since not only are the dynamics of the environment stochastic but also because the underlying stochastic mechanism that generates these dynamics are unknown.   
%
%In the context of security games, the random selection of passengers for security checks at an airport is an example source of uncertainty in this game where the outcome is random and the probability of successful security enforcement  is unknown. 
%In these applications, a security game solver is required to autonomously take intelligent decisions at repeated instances of the game, and must thus carefully gather, extract and act upon information from historical data. 
%
\paragraph{Bandit problems}
Mathematical machine learning is a modern amalgamation of statistics (to deal with uncertain data) and optimisation (to efficiently select appropriate actions).   	
%It has proven to make a difference in practice and has nice mathematical background. We want to focus here on two particular methods that are interesting in machine learning and where Victor and David are experts.
%First, many interesting sequential decision-making tasks can be formulated as reinforcement learning (RL) problems. In RL, an agent interacts with a dynamic, stochastic, and incompletely known environment with the goal of learning a strategy or \textit{policy} to optimize some measure of its long-term performance (e.g.,~to remove as many lines as possible in Tetris).
%talk very quickly about MDP POMDP tree search?
One of the fundamental problems in machine learning, relevant to our research objectives in this proposal, is the \textit{multi-armed bandit problem}. It corresponds to a scenario where the learner is required to actively collect data from an environment in order to solve a given task. Solutions built for this problem have found many practical applications from adaptive routing in a network to medical trials of new medicines\cite{bubeck2012regret}.
A multi-armed bandit is precisely a game as described above, but where the attacker has only one action; the bandit problem results when the game is repeated, and on iteration $t$ the learner selects action $i(t)$ and received a reward $l_t$ which is a random variable with expectation equal to $\boldsymbol A_{i_t,1}$.
%In its simplest form there are $K$ actions (arms). At each round $t$, the environment allocates rewards to each arm (described as a vector $l^t\in\R^k$) while the learner simultaneously chooses an arm $I(t)$ to pull in order to obtain a reward $l^t_{I(t)}$.
An important constraint in this setting is that the player is not allowed to observe the hypothetical reward that would have been collected had another arm been selected instead.

In the bandit problem two related tasks are commonly considered.  One is the {\em online decision-making} task, in which the reward for each decision must be taken into account; this task is relevant to the immediate deployment of the system to actually make decisions while it learns.  The standard performance metric here is the regret, defined to be the difference between the total reward that could have been achieved if full information were available in advance, and the actual reward that was achieved.  The other task is that of {\em pure exploration}, in which a learning phase is permitted during which received rewards do not matter; this corresponds to allowing a training phase for the system prior to deployment.  The performance metric of this problem is the quality of the arm selected immediately after the training phase.  Application of the pure-exploration problem to parallel action selection in robotic planning has been extensively studied by Dr Gabillon \cite{Gabillon11MB}. 

When we consider more general security games, in which the attacker has more than one available action, we can instead think of a non-strategic attacker who uses a fixed distribution over actions through time.  This also corresponds to bandit problem for the attacker.
Bandit strategies are therefore important for active learning in security games.  However it is as yet an important open question how to take into account strategic or adaptive behaviour of the attacker.  It is this aspect of learning in security games which will be developed in the proposed project.
%This therefore corresponds to a simplified security games where the strategies of the other players are fixed and only an external environment allocates the rewards. One can think of a game where the $K$ arms/actions correspond to the $K$ possible security strategies whose values are determined according to some unknown underlying probability distribution. 
%The average per action reward can be estimated through sampling, i.e. via pulling the corresponding arm. 
%The problem has been studied under two main assumptions about the rewards. 
  
%all of the utilities are chosen by an  the defender takes actions against an attacker who adversarially designs the rewards of the game.
%For this proposal we focus on two different ways to measure the performance of the methods that corresponds to two different security requirements. % The first formulation corresponds to the classical cumulative regret setting where the forecaster tries to constantly choose the security strategy with the highest value on average.  The second one is the ``pure exploration'' setting, where the forecaster can uses the exploration phase, a limited phase during which he can freely pull the arms he chooses to, in order to identify the best  security strategy among the $K$.

%Sequential decision-making tasks is often coupled with online learning in the sense such problem needs to learn online how to solve a task while solving it. These two features, sequential decision making and online learning while be most required for our new problems right! elaborate a bit


%The problems put forward a notion of complexity of the problem. That we try to characterise well. And that we will try to study in our new problems.

%While the multi-arm bandit games are extremely indispensable to active learning, an important question is how they can be used in general security games where one has to also take into account the actions of the other players.


%\noindent \textbf{\textit{\\Frontiers of Security Games: From handling uncertainty towards self-learning algorithms}}\\
\paragraph{Existing results}
%An extensively litterature has tried to  devise security games methods that are robust with respect to uncertainty about the environment\cite{aghassi2006robust,Nguyen14RO, Kiekintveld:2013}.
%Granick, for example,
%argues that weaknesses in our understanding of the measurability of losses serve as
%an impediment in sentencing cybercrime offenders\cite{granick2005faking}. Swire adds that deterring
%fraudsters and criminals online is hampered if we cannot correctly aggregate their
%offences across different jurisdictions \cite{swire2009no}. These  interest in dealing with uncertainty in the data shows how important and crucial and costly it could be and that it is a real issue in real problems.
Recent advances have been made in this direction, which mostly focus on the case where the attacker's preferences are not fully known and must be learned through repeated plays of the game. One approach is to analyse the number of required queries to learn the optimal defender's strategy \cite{blum2014learning, letchford2009learning}. 
Alternatively a Bayesian approach can be taken, with techniques based on Partially Observable Markov Decision Processes (POMDPs) used to update a posterior over the adversary's preferences\cite{Marecki12PR, qian2014online}. 
%The main theoretical drawback of this planning method is in that the algorithm is based on Upper Confidence Trees (UCT), which, are provably sub-optimal\cite{munos2014bandits}. 
Recently, a more relevant analysis has been given\cite{Balcan15CR} for the case of multiple attackers, where at each round of the game, a single attacker is chosen adversarially from a fixed, finite, set of known attackers. This corresponds to a case where the utility matrix $\boldsymbol B$ is chosen adversarially from a set of $k$ known matrices, and shows strong connections with adversarial bandit theory.  However all of these security games results rely restrictive assumptions about prior knowledge and observability, and on the number of available actions to each player being reasonably small.  In this proposal we consider combinatorial actions (selecting $k$ resources to protect out of $n$ that may be attacked) and so the total number of actions available is enormous.

%\textbf{Motivation.} %%%V what can I bring

\paragraph{Vision}
%\noindent \textit{\textbf{\\Main Goal}}\\
%The main issues that have been considered so far in the literature are scalability, the ability for the designed methods to handle problems with very large number of actions for the players, or devising strategies that take advantage of the attacker's potentially limited rationality or bounded memory\cite{tambe2012game}.
The purpose of this project is to create \textit{practical}, \textit{scalable} and \textit{robust} methods for security games. First, we target \textit{practicality}  in the sense that our methods will be autonomous in handling uncertainty in the model and will actively reduce this uncertainty by interacting with the environment in which the game takes place during repeated plays of the game. We will do so by combining existing security game research with the extremely active field of multi-armed bandit research.
%Specifically, we aim to broaden the scope of repeated security game problems where the initial uncertainty about the players utilities can be overcome through using learning techniques  in conjunction with  repetitive plays of the game. 
%These techniques are from the extremely active field of machine learning research.
Second, we target \textit{scalability} so that the methods will apply with an extremely large number of possible actions. This can be achieved by making simplifying structure assumptions such as combinatorial structure, in which an action consists of selecting $k$ objects from $n$. % will be considered as well as the simplifying submodular property of the objective function.
Finally \textit{robustness}  is a key issue in security games in several senses.  Methods should not break down in the face of adversarial play by the attacker both during learning (so that the defender receives the least helpful information from which to learn).  Furthermore methods should not perform well only in average --- worst case performance is extremely important.  We will therefore devise a theoretically sound approach in which efficient algorithms are developed and finite time performance guarantees are provided.
%We propose to ensure robustness in  three different ways. First, one can be  very conservative and assume that an adversary actually chooses (in the most adversarial way) the data that we do not know. Second, we introduce offline learning  where the test and the learning happen  before the strategies is actually use in the real world. Third, we might be required to learn defence strategies that do not possess large variances in their performance.

%\textit{Efficient learning algorithms:} 
%Our goal is to have a theoretically sound approach by designing efficient algorithms for which we can provide finite sample analysis.%First we will look at new instances of these problem that are related to our area of bandit expertise and that serve real purpose.
%To solve real world problems our \textit{long-term} goals are:

%\begin{itemize}
%\item \textit{Explore stochastic and adversarial assumptions:} In a first setting, the environment chooses the rewards stochastically, based on some predetermined (however unknown) distributions. In security games this stochastic assumption helps model the unknown dynamics of the problem which both the defender and attacker can learn through interaction with the environment. In the second setting, no stochastic assumptions are made. Instead, an adversary is assumed to have arbitrarily chosen and designed the rewards
%\cite{Auer03NS}. This can be used to model a more pessimistic security game where the utilities are chosen adversarially by the attacker.
%Aside from contributing to adversarial setting, one of our interest  is to make Stochastic assumptions when dealing with noise in the model and adversarial assumption when dealing with the adversary to make our approach both realistic and robust.
%%%A depending on the considered scenraio, it may make sense to consider the problem as adversarial or stochastic. The distinction will be made clear in the ... 
%So far stochastic noise in the model  in conjunction with learning has been untouched while it can happen when phenomenon are not inherently adversarial (sensor that work stochastically action that have stochastic outcomes, checkpoint might not stop deterministically the attacks and the probability of success needs to be determined, here learned).
%Mention my previous work?




%The \textit{short-term} plan in achieving these goals is outlined in three main objectives below.




\textbf{Objective 1: Scalability in Pure exploration Bandits via Submodularity.}
When considering the combinatorial nature of many security games, it is necessary to use this structure intelligently.  In particular, naively enumerating all possible actions as in the standard formulation of security games makes the computations rapidly intractable.  
%As discussed previously, bandit problems extend naturally to repeated games under uncertainty. 
We will address this problem by first studying combinatorial bandit techniques, which will subsequently be extended to use in security games.  Combinatorial bandits form a central part of Dr Gabillon's area of expertise.   A key observation is that in many cases the players' performance utility function is submodular. One example is in the context of maximal coverage problem for sensor (or checkpoint) placement\cite{krause2011randomized}. 
 This submodularity property can in turn be used to provide tractable and almost optimal algorithms in the online decision-making settings of bandits\cite{gabillon2013adaptive}.  However pure exploration has not yet been solved in this area; Objective 1 is therefore to complete the analysis of combinatorial bandits by addressing the pure exploration problem under submodularity assumptions.
% 
% As discussed further in the objective 2, we are also interested the bandit formulation called pure exploration that will provide a safer scenario to tests the security strategies. As this bandit setting is part of the fellow expertise a very natural first step will be to consider combinatorial pure exploration under submodularity.
 
%  A numerous literature have been written on extending bandit approach to linear, combinatorial or continuous spaces. In this proposal we wil be more intereted in combinatorial assumption that aris ein security games  (games in networks, combinatorial actions). 
%HoweverRecently the fellow has proposed
 
 
\textbf{Objective 2: Pure Exploration in Security Games.}
%Contrary to the cumulative regret setting, the rewards collected before the end of the game are not taken into account. 
As discussed previously, a major barrier to applying security games to real-world scenarios is that the players' utility matrices $\boldsymbol A$ and $\boldsymbol B$ are unknown and must be learned. In the first of two objectives on standard security games we consider the case where the defender is in fact able to safely examine her defensive strategies before applying them online, corresponding to the pure exploration bandit framework. A familiar real-world example is the security system at an airport, which can be tested many times before being deployed as the principal defence scheme.  
%This corresponds to best-arm identification in the bandit problem.
%
%We observe that in this formulation of the problem, we can capture the learning module of the security game by a recent bandit setting called \textit{pure exploration}, where the learner is only evaluated at the end of an exploration phase comprised of a limited number of interactions with the environment. Pure exploration has generally proved useful in many practical scenarios. Specifically, its application to parallel action selection in robotic planning has been extensively studied by the fellow \cite{Gabillon11MB}. 
%However, applying this approach to security games is an open research path with a strong potential to bring new and interesting angles to the domain. 
In this formulation, we assume both utility matrices $\boldsymbol A$ and $\boldsymbol B$ unknown. During the test phase, the defender is able to probe an entry of its utility matrix at every mock repetition of the game. The value obtained is {\em a noisy version} of the true entry $\boldsymbol A_{i,j}$. 
%What makes this setting particularly well-suited to real-world security applications is in that, 
%1. unlike existing work, no assumption is required to be made about the defender's knowledge of its utility matrix $\boldsymbol A$. Instead, the utilities can be learned offline through probing the individual entries of $\boldsymbol A$, and 
%2. the stochastic framework gives a natural model for the players' uncertainty about the environment.
%
 
% \textbf{Complexity.}
 We will design a strategy for the defender to either minimise the number of tests needed to identify an excellent strategy with a given level of confidence\cite{Maron93HR,Even-Dar06AE} or to maximise her probability of identifying the best strategy given a fixed number of tests\cite{Bubeck09PE,Audibert10BA}. 
 %In the classical multi-armed bandit problem, these optimisation objectives respectively correspond to a \textit{fixed confidence} setting and a \textit{fixed budget} constraint .
 In order to extend these  classical results to the setting proposed above, we will first carefully characterise the data-dependent hardness of the problem, extending recent relevant results for combinatorial bandits\cite{chen2014combinatorial} and similar results currently in preparation by Dr Gabillon.  Of course, in games these complexity results are more challenging than in the bandit problem since the complexities depend also on the actions available to the attacker.
%  and then move on to designing algorithms to best capture the obtained hardness. In standard bandit, the complexity of an arm is inversely proportional to the gap $\Delta_i= \mu^*- \mu_i$ between the value of the best option $\mu^*$ and that of option $i$. Extensions of this notion have been designed for combinatorial bandits \cite{chen2014combinatorial}. The fellow is currently working on a improved version of the state-of-the-art result. The complexity of an arm defined in these combinatorial games is  more complex than in the simple multi-armed bandit problem as it involves combinatorial quantities. Similarly in security games, we expect the complexities of each entry of the matrix for the defender to depend also on the actions available to the attacker.
  Therefore we will study this problem by gradually increasing its difficulty with different partial feedback structures. First we note that a recent work\cite{goldberg2014query} on query complexity, corresponds to the simpler deterministic  version of this problem where it is assumed that the probing outcome corresponds to the {\em true value} of $\boldsymbol A_{i,j}$. %Thus they do not analyse the number of queries required to correctly estimate each entry of the utility matrix, which in turn depends on the performance of the strategies that would use this entry. 
Therefore our first approach will be to combine ideas from pure exploration and the deterministic query complexity setting in a context where the defender can individually sample from any entry of the matrix. A second more challenging setting will be to consider the more adversarial learning problem where the defender chooses a strategy and only observes the value of the game corresponding to an action selected by the attacker; the attacker might be either oblivious to the defender, or playing a Stackelberg best-response to the defender strategy.
 %Note that the hardness of the best arm identification problem in the stochastic setting can be interpreted as the total number of pulls required to discriminate the best arm(s) from the others. In a simple multi-arm bandit setting, this is defined as the sum of the complexity of each suboptimal option, where the complexity of a suboptimal option $i$ is inversely proportional to the gap $\Delta_i= \mu^*- \mu_i$ between the value of the best option $\mu^*$ and that of option $i$.
%More precisely the complexity $H$ is  defined as $H = \sum_{k} \frac{1}{\Delta_k^2}.$ 


%\textit{Different feedback structures:} The complexity of learning highly depends on the quality of the feedback that the learned can collect. From the most informative full information feedback to the many variations of partial feedback setting it is of importance, as discussed in Objective 1 below, to quantify how the nature of the feedback  affects the performance and to focus on scenario that correspond to real world examples.

 %
 %submodularity property of the problem that often arises in the context of sensors (or checkpoints) placement can provide tractable and almost optimal algorithms. Therefore submodular combinatorial pure exploration is a first step.



%One first step is to relax the problem as shown in Krause et al finding the best response to a given adversary. This is known to be is NP hard problem  but can be solve almost optimally be a greedy algorithm thank to a sub modularity property of the problem. This gives rise to a first objective which would be to learning optimise stochastic submodular function under a pure exploration setting.
%Note that I worked on similar subject with learning in submodular functions.



%The idea would be to  extend the work of Balcan using more complex bandit algorithms. They use a version with k known attackers. We can assume that k is extremely large but there is some  structure that permits us to use for instance combinatorial bandits.

%The security issue naturally has application in graph problem that model the network of roads/ connection between computers that agents might need to secure. Therefore there has been study that apply game theory to this problems. For instance it has been used to monitor road barrage in mumbai (connection) The goal is there to put some check point on a road to stop some terrorist. Its a one shot game where you try to minimise the probability of the player to pass.  Utilities are not really defined and complex here You just want to maximise the probability of catching the attacker. We are interested in a version of this game that is repeated . Everyday the same problem arises. We would minimise the cumulative regret. Therefore the defender can be adaptive and if the attacker is not smart and repeat always the same plan we will catch him often (not totally a worse case scenario). This can be seen actually has a specific problem of adversarial combinatorial bandits where the  attacker is limited to a very specific structure of losses which are path in a graph. We can expect to use the specificity of the graph by using some result from spectral graph theory. Maybe also we can use this theory to solve some issue with the scalability of the algorithm.

\textbf{Objective 3: Regret Analysis of Repeated Security Games.}
In some applications a newly created security system is not provided with any historical data and cannot be tested before being used in production. Here, the learning of the utilities must be performed online while actually playing the security game. In this online decision-making context it is of high importance for the agent to learn the utilities as fast as possible. This means that only providing an analysis to demonstrate asymptotic convergence\cite{LeslieCollins06,ChapmanEtAl2013} is inadequate. To address repetitive learning in security games we will here assume that the utility matrix $\boldsymbol  A$ is unknown to the defender and that, at each repetition of the game the attacker will best respond to the defender's current strategy (if $\pi$ is a mixed strategy of the defender, then denote $b(\pi)$ the best response of the attacker). Therefore we are considering a setting that is related to the analysis of Stackelberg equilibrium. A natural quantity of interest is the \textit{cumulative regret}, defined as
\begin{equation*}
R(n)=n\max_{\pi} \left[\pi A b(\pi)\right]-\sum_{t=1}^n \pi_t A b(\pi_t).
\end{equation*}
This compares the actual reward received (assuming the attacker always performs as well as they can) to the reward achieved at Stackelberg equilibrium (when the defender chooses the best possible mixed strategy under the knowledge that the attacker will best respond to it).  The objective is for the defender to build a series of defence strategies $\pi_t$ for $t=1,\ldots, n$ to minimize the expected cumulative regret $R(n)$.
% 
%
%where the v(G) is the value of the game when the two players play according to the Stackelberg equilibrium of the game. The cumulative regret would be therefore defined the difference between the sum of rewards collected by always using the \textit{best security strategy} in hindsight and the sum of rewards actually collected by the forecaster.
% This is an online game where, at each time set $t$ of the game, a fundamental trade-off arises for the forecaster between the simultaneous need to select the security strategy solution he currently thinks is the best in order to maximise the immediate security given his current knowledge  (exploitation)  while also wanting to test possible other strategies that might or might not be better (exploration).
 
 This game-theoretic scenario, is actually strongly related to the bandit problem, in that the best-responding assumption on the attacker leads to a situation where the reward to the defender depends only on the selected (mixed) strategy.  A solution can thus be obtained by considering it as a bandit problem with continuous action space consisting of the set of all probability distributions on the original discrete action space.  However a more efficient solution is likely to be obtained by explicitly considering the game-theoretical nature.
%
%Our goal here will be to provide a regret analysis for the online stochastic repeated security games. The first step will be to study  how to design a new algorithms borrowing ideas from the UCRL algorithm \cite{auer2009near} that was extending the bandits ideas of the classical UCB analysis to general reinforcement learning problems. 
In particular, most current approaches for online decision-making in bandits, such as upper confidence bound methods\cite{Auer02FA}, implement a strategy that is optimistic in face of uncertainty, playing any strategy that could be the best given the level of uncertainty. It will be extremely interesting to discover if this optimism principle still holds in an adversarial game, or whether a more cautious approach is needed.
%especially in a case where the adversary knows the true utilities of the games and moreover knows your levels of uncertainty.
%For this project collaboration with  Bruno Scherrer, a fellow's co-author, working at  Inria Nancy, would prove fruitful as he has recently analysis how reinforcement learning asymptotic analysis could be applied to zero sum games\cite{scherrerapproximate}.
Approaches based on Thompson sampling\cite{russo2014information} will be also considered as they have proved very efficient in practice and correspond to an area of expertise of Prof. Leslie.
%talk about the UCT stuff?

Finally an interesting additional requirement is to learn security strategies that are not only of good quality in average but also whose performance is not subject to large variance when used on a daily basis. This {\em risk-averse} requirement has been well-studied in the statistical community, % has a \textit{risk-averse} requirement. Recently this requirement 
and has recently been considered in a multi-armed bandit framework\cite{NIPS2012_4753}. An implementation in the security games specific context is a very natural extension to the main body of work in this objective.




\textbf{Objective 4:  Learning in combinatorial games.}
Objective 4 will be devoted to solving security games with more complex action structures. 
Real-world security problems often involve large, complex networks. This includes, for instance, complex routes, or computer/communication networks. The size of the action spaces for both attacker and defender is often combinatorially large.  Standard results with convergence times that increase with the size of the action spaces become extremely weak in such settings.
We will therefore take advantage of the inherent combinatorial structure of the problem to create efficient and computationally tractable algorithms in these large games.   In particular, we will develop the approaches of both Balcan\cite{Balcan15CR} in simple Stackelberg games and  in adversarial combinatorial bandits\cite{cesa2012combinatorial}.
The issue of scalability will be addressed in light of the results found in Objective 1.

\textbf{Objective 5:  Repeated Network-Security Games.}
As a more concrete application of Objective 4, this objective will focus on the particular combinatorial structure that is a graph as this stucture is present in  numerous real-word applications.
In light of the ever-growing, modern, social and communication networks, a canonical example is that of smuggler arrest in a network\cite{jain2011double}. This has received significant attention in the community, especially in  response  to  the  Mumbai  attacks  of  2008, after which  Mumbai  Police
started to schedule a limited number of inspection checkpoints
on the road throughout the city.
This problem has not yet been studied in its repeated form, where a pursuit-evasion game is played multiple times against a population of smugglers. Therefore, the currently deployed strategy of the defender is \textit{not adaptive} to observations collected about the attackers' historical strategy and is therefore sub-optimal.  We will therefore develop adaptive strategies, using the approaches developed in Objectives 1--4.  However, the graph structure provides additional constraints on the action spaces, and provides additional information, when compared with the general combinatorial problem.  In particular, the set of actions available to the attacker is restricted to a set of paths through the network, and the the graph structure provides strong information about sensible choices of checkpoints (for example, aligning them all along one route through the network is a particularly unfortunate choice, but is not ruled out by a generic combinatorial structure).  Furthermore, absence or presence of a smuggler on one link of the graph will likely provide information about which other other links were utilised on that iteration. Therefore the objective  is to design specific algorithms in in situations where it is possible to take advantage of specific graphical structure of the problem. We will start by defining a notion that captures the hardness of the task depending on characteristics of the graph that we would discover. Note that, although the goal is to generate algorithms for security games on graphs, the results to be obtained will be expected to lay grounds for research in a more general setting of active learning with graph structure. Dr Gabillon has held initial discussions on this topic with Dr Michal Valko, a world-famous expert in active learning on graphs and part of INRIA Lille in France. This project will allow the formation of a productive and lasting collaboration with Dr Valko, which will be greatly beneficial not only in achieving the this objective, but also to strengthen international links between Lancaster University in the UK and INRIA in France.  
 
 
 %A popular efficient  algorithm for it is Thompson Sampling as it has both proved very efficient\cite{Chapelle11EE}, handy and as recently started to be theoretically good~\mbox{\cite{Kaufmann12TS}}. 

%Note that a very numerous of variation of the initial games ranging from considering  infinite number of arms to continuous actions\cite{Wang08AI,Abbasi-Yadkori11IA,Dani07TP} permits these methods to adapt to a very large mount of challenges

%\noindent \textbf{\textit{\\Originality and innovative aspects of the research programme:}}\\
%This grant proposal is original has it proposes to develop a bridge between two areas of research (Security games and Machine learning) that have both proved their practical capacities but have only rarely been used jointly. By addressing some of the fundamental challenges of this connection we expect to bring more attention to the potential of this connection and bring the two communities to collaborate more in order to design software that solve important security problems.
%Our approach is to design algorithms are both theoretically grounded and of practical use in real world problems.
%Moreover by collaborating with with the Security Lancaster Departement in the University of Lancaster, we hope to apply and test our methods to real world problems.
% 
%\noindent \textbf{\textit{\\Timeliness and relevance:}}\\
%Security games has found numerous applications in the recent 10 years and has proved that it could bring a significant improvement over  systems designed by humans. Making this systems even more autonomous so that they can autonomously improve by constantly learning is a key issue to bring low maintenance security systems. We believe machine learning is providing most of the necessary tool to reach this goal. Machine Learning is one of the most rapidly growing community in computer science research as it is currently making a difference in key industrial sectors already  building strong recommendation systems, and being used by the most important companies of our time to create self-driving cars or new pattern recognition algorithms.   The connection between security games and machine learning is evident, work that have connects those two worlds are still rare and it will be the source of many new research challenges. 
%Finally bringing more robust security systems is of high importance for Europe has recent event have shown how ensuring the security of network communications from spy or securing public transportation is crucial.
% 



%Or say that the approach that do only rarely propose strong theoretical results (to the notable exception of Blum)
%like \cite{Marecki12PR} (which by the way use a theoretically not sound algorithm UCT) 
\subsection{Clarity and quality of transfer of knowledge/training for the development of the researcher in light of the research objectives}
\label{sec:transfer}

{\em Outline how a two way transfer of knowledge will occur between the researcher and the host institution, in view of their future development and past experience: (please see Section 5.2 of this Guide):
\begin{itemize}
\item Explain how the Experienced Researcher will gain new knowledge during the fellowship at the hosting organisation(s)
\item Outline the previously acquired knowledge and skills that the researcher will transfer to the host organisation
\end{itemize}
}

The overall trianing objective is to significantly develop Dr Gabillon's scientific, organisational, communication and technology transfer skills.  This will enable him to continue building his portfolio of outstanding research to attain a position of independence and gain recognition in the international research community.

The proposed project is primarily a research project, and the main training objectives are to enhance the fellow's scientific skills. Dr Gabillon is already an expert in the modern theory of bandits, including best arm identification, and reinforcement learning.  Therefore this project's main training objective for Dr Gabillon will be to develop his skills and knowledge in statistical learning methods and game theory.  The supervisor is an expert in both areas, and will of course assist the development of the Researcher.  Further expertise in Lancaster from whom the Researcher will learn includes the Statistical Learning group, in which the Researcher will be based, and the broader Statistics Research Group.

Considering also the research group in operations research within the Management School, Lancaster is the leading UK institution in bandit theory, with expertise in index policies (Glazebrook, Kirkbride, Jacko), Thompson sampling and contextual bandits (Grunewalder, Leslie) and application in medical trials (Vilar). Dr Gabillon will have ample opportunity to further develop his expertise in this area, and indeed brings expertise from a complementary aspect of online learning and decision-making in the design and analysis of algorithmic approaches to learning, especially with combinatorial bandit problems.   Dr Gabillon's expertise in best-arm identification will be of great interest to the Medical and Pharmaceutical Statistics research group.  He will present his research in this area to the research group and discuss possible applications in clinical trial design.  Furthermore his expertise in combinatorial bandits complements current industrially-funded research of the Supervisor.

In addition to his research skills, Dr Gabillon will learn from the host's world-leading expertise in developing industrially-inspired statistics.  Statistical researchers in Lancaster have constant exposure to external companies, through the STOR-i Centre for Doctoral Training, and the Data Science Institute.  While embedded in this culture, Dr Gabillon will be given the opportunity to:
\begin{enumerate}
\item Gain further experience of developing industry/academic partnerships by working with Profs.\ Leslie and Eckley and other staff in STOR-i and the Data Science Institute in technology transfer activities.
\item Develop public communication skills by presenting research results to varied audiences.
\item Participate in the organisation of workshops in Lancaster and at the Royal Statistical Society.
\item Receive training on preparing funding applications by co-authoring proposals for UK and EU funding agencies with Prof.\ Leslie and others.
\item Attend staff training workshops designed specifically for early-career researchers, and specifically the Research Development Programme, a structured development route for researchers, designed to promote impactful research and to support development beyond a disciplinary area.
\item Participate in teaching and research supervision at undergraduate and graduate level.  This will not be obligatory, but the fellow will be given the opportunity to benefit from peer observation, mentoring, and constructive criticism.
\end{enumerate}

Throughout the fellowship, Dr Gabillon will adhere to the ``European Charter for Researchers'', and the training objectives will be managed through a Personal Career Development Plan that Prof.\ Leslie and Dr Gabillon will write together.  This plan will be revised regularly throughout the fellowship to ensure that all objectives are met.  In addition, Dr Gabillon will have regular meetings with the host supervisor to discuss his research and to receive advice.


\subsection{Quality of the supervision and the hosting arrangements}
\label{sec:supervision}

{\em
Required sub-heading:
\subsubsection*{Qualifications and experience of the supervisor(s)}

Information regarding the supervisor(s) must include the level of experience on the research topic proposed and document its track record of work, including the main international collaborations. Information provided should include participation in projects, publications, patents and any other relevant results.
To avoid duplication, the role and profile of the supervisor(s) should only be listed in the "Capacity of the Participating Organisations" tables (see section 6 below). 

\subsubsection*{Hosting arrangements\footnote{The hosting arrangements refer to the integration of the Researcher to his new environment in the premises of the Host. It does not refer to the infrastructure of the Host as described in Criterion Implementation.}}
The text must show that the Experienced Researcher should be well integrated within the hosting organisation(s) in order that all parties gain the maximum knowledge and skills from the fellowship. The nature and the quality of the research group/environment as a whole should be outlined, together with the measures taken to integrate the researcher in the different areas of expertise, disciplines, and international networking opportunities that the host could offer.

For GF both phases should be described - for the outgoing phase, specify the practical arrangements in place to host a researcher coming from another country, and for the incoming phase specify the measures planned for the successful (re-)integration of the researcher.

Describe briefly how the host will contribute to the advancement of their career. In that context the following section of the European Charter for Researchers refers specifically to career development:
}
%\fbox{\begin{minipage}{\textwidth}\paragraph{Career development}
%Employers and/or funders of researchers should draw up, preferably within the framework of their human resources management, a specific career development strategy for researchers at all stages of their career, regardless of their contractual situation, including for researchers on fixed-term contracts. It should include the availability of mentors involved in providing support and guidance for the personal and professional development of researchers, thus motivating them and contributing to reducing any insecurity in their professional future. All researchers should be made familiar with such provisions and arrangements.\end{minipage}}
%
%Therefore a Career Development Plan should not be included in the proposal, but it is part of implementing the project in line with the European Charter for Researchers.

\subsubsection*{Qualifications and experience of the supervisor(s)}

Prof.\ Leslie leads the Statistical Learning research group in the Department of Mathematics and Statistics, Lancaster University.  He is a world-leading researcher in statistical learning, Bayesian inference, decision-making and game theory, with 19 refereed articles in top journals of several different research fields, and collaborators from France, USA and Australia.  His research on contextual bandit algorithms \cite{MayEtAl2012} is used by many of the world's largest companies to balance exploration and exploitation in real-time website optimisation.  He is expert in the mathematics of learning in games, \cite{LeslieCollins03,LeslieCollins05,LeslieCollins06,ChapmanEtAl2013,PerkinsLeslie2014} stochastic approximation, \cite{LeslieCollins03,PerkinsLeslie2012,PerkinsLeslie2014} and the mathematics of statistically-inspired reinforcement learning. \cite{LeslieCollins05,LarsenEtAl2010}  Prof.\ Leslie is the holder of a Google Faculty Award which funds a student to investigate multiple-action selection in bandits.  Prior to his relocation to Lancaster, he was a senior lecturer in the statistics group of the School of Mathematics, University of Bristol.  He continues to be co-director of the \pounds1.5m EPSRC-funded cross-disciplinary decision-making research group at the University of Bristol, and was on the management team of the \pounds5.5m ALADDIN project, a large strategic partnership between BAE Systems and EPSRC, involving researchers from Imperial College, Southampton, Oxford, Bristol and BAE Systems.

Prof.\ Leslie's mentoring approach is one of `guided freedom' in which the mentee takes responsibility for their own research, while regular discussions ensure that dead ends are avoided and promising openings are exploited.  In the 10 years since taking up a Faculty position, he has supervised 17 PhD students, 2 post-doctoral fellows, numerous MSc and undergraduate dissertations, and an undergraduate secondment from ENS Lyon.

\subsubsection*{Hosting arrangements}

Dr Gabillon will be embedded within the statistical learning group which is lead by Prof.\ Leslie.  This is a team of 5 academic staff and around 5 PhD students within the Department of Mathematics and Statistics.  The Researcher will participate in weekly group meetings and benefit from advice from the senior scientists in the group, including the Supervisor, on research direction and management, personal development, workshop organisation, teaching, and other aspects of academic life.  The group also has extremely strong links with both the Data Science Institute (www.lancaster.ac.uk/dsi/) and the STOR-i Centre for Doctoral Training (www.stor-i.lancs.ac.uk/), each of which have approximately weekly seminars.  These exciting initiatives will provide multiple further opportunities to develop informal mentoring relationships in addition to the formal process which takes place for all staff at Lancaster University; to ensure integration within these networks the Researcher will be introduced to the groupings of researchers, invited to deliver a seminar on his research, and will participate on project away days in which strong relationships are developed.


\subsection{Capacity of the researcher to reach and re-enforce a position of professional maturity in research}
\label{sec:maturity}

{\em
Applicants should demonstrate how their proposed research and personal experience can contribute to their professional development as an independent/mature researcher.

Please keep in mind that the fellowships will be awarded to the most talented researchers as shown by the proposed research and their track record (Curriculum Vitae, section 4), in relation to their level of experience.
}

\TODO{Victor to have a first stab}

\section{Impact}
\label{sec:impact}

\TODO{Demonstrate: worthwhile outreach, good communication strategy (are there existing connections that can be exploited?), adequate discussion of impact on researcher's career, indication of how outreach activities will be assessed, strategies for exploitation of outcomes.}

\subsection{Enhancing research- and innovation-related skills and working conditions to realise the potential of individuals and to provide new career perspectives}
\label{sec:enhancement}

{\em
Explain the expected impact of the planned research and training, and new competences acquired during the fellowship on the capacity to increase career prospects for the Experienced Researcher after this fellowship finishes.

Demonstrate also to what extent competences acquired during the fellowship, including any secondments will increase the impact of the researcher?s future activity on European society, including the science base and/or the economy
}

Dr Gabillon is already a lading researcher in the mathematics of bandit algorithms and reinforcement learning.  This fellowship provides a training oppotunity in two key additional research competences.  Firstly, the Researcher will develop an in depth knowledge of cutting edge statistical theory, and bring that to bear within bandit algorithms.  Training will be received from leading scientists in statistics and operations research at Lancaster University, and the many international visiting researchers who visit the department.   Secondly, the Supervisor is a leading expert on learning in games, as well as bandit algorithms, and will mentor the Researcher to bring ideas from bandits into the game theoretical scenarios of this research proposal.  This significant broadening of the researcher's skill set will give him an extremely solid foundation on which to build a future research career.

In addition to pure research opportunities, Dr Gabillon will work with Lancaster University's extremely effective mechanisms for industrial collaboration.  He will develop skills in how to manage the industry/academia relationship to ensure mutually beneficial outcomes.  This relationship-management will be a skill for academics in the future; Lancaster University, and particularly the Department of Mathematics and Statistics, is currently a world-leading institution in developing such relationships.  The Researcher will both be introduced to prospective industrial partners, and receive mentoring as he develops his own relationships.


\subsection{Effectiveness of the proposed measures for communication and results dissemination}

{\em
The new knowledge generated by the action should be used wherever possible to advance research, to foster innovation, and to promote the research profession to the public. Therefore develop following three points.
\begin{itemize}
\item Communication and public engagement strategy of the action
\item Dissemination of the research results
\item Exploitation of results and intellectual property rights
\end{itemize}
Concrete plans for the above must be included in the Gantt Chart (see point 3.1).
The following sections of the European Charter for Researchers refer specifically to public engagement and dissemination:

\fbox{\begin{minipage}{\textwidth}
\paragraph{Public engagement}
Researchers should ensure that their research activities are made known to society at large in such a way that they can be understood by non-specialists, thereby improving the public's understanding of science. Direct engagement with the public will help researchers to better understand public interest in priorities for science and technology and also the public's concerns.
\paragraph{Dissemination, exploitation of results}
All researchers should ensure, in compliance with their contractual arrangements, that the results of their research are disseminated and exploited, e.g. communicated, transferred into other research settings or, if appropriate, commercialised. Senior researchers, in particular, are expected to take a lead in ensuring that research is fruitful and that results are either exploited commercially or made accessible to the public (or both) whenever the opportunity arises.\end{minipage}}
}

%\TODO{Think about public engagement. I'm planning to set up a ``Data Science Network'' around Lancaster to help generate both enthusiasm and contacts within local companies.  Now might be a good time to write something more formally about it!}

With the launch of the Data Science Institute, Lancaster University will be inaugurating a ``Data Science Network'', bringing together academic data scientists with local companies in regular show and tell sessions.  The Researcher will be a regular participant at these events, enabling him to present aspects of his research to the local business community, develop an understanding of the business requirements for this kind of user, and build a network of industry contacts.  In addition, Lancaster University supports researchers to write for the Conversation, a news  service delivering articles directly from researchers to the public; the Researcher will make use of this support to produce expository articles explaining the benefits that adaptive data science approaches can deliver to society.

The excellent and innovative research generated in this project will of course be published Open Access in the world's leading academic journals and conferences.  Prof.\ Leslie currently works with several companies, both large and small, and Dr Gabillon will be mentored to develop similar relationships.  We will discuss results directly with companies in Lancaster University's Knowledge Business Centre, an innovation hub providing a gateway for business/academic interaction which allows the transfer of expertise between Lancaster's academics, regional businesses and community partnerships through training and technology transfer activities.  A particularly successful mechanism deployed extensively at Lancaster is the industrially-sponsored MSc or PhD project, which allows the supervisor's research to be both developed and deployed directly within a company; the Researcher will be encouraged to join appropriate supervisory teams to help both disseminate the project's research and develop an industrial research network to enhance his future career.  The Research Support Office of Lancaster University has extensive experience of industrial engagement and will assist in the management of IP and any patents that may arise from the research.

\section{Implementation}
\label{sec:implementation}

\TODO{Show them: specific tasks and clearly-defined outputs/deliverables; host institution has capacity to support researcher; coherent workplan (including justification for the scheduling); metrics to assess progress; clear management structure (ie what is done beyond regular supervisor meetings); risk management and contingency plans; quality management procedures}

\subsection{Overall coherence and effectiveness of the work plan, including appropriateness of the allocation of tasks and resources}

{\em
Describe the different work packages. The proposal should be designed in such a way to achieve the desired impact. A Gantt Chart should be included in the text listing the following:
\begin{itemize}
\item Work Packages titles (for EF there should be at least 1 WP);
\item List of major deliverables;\footnote{A deliverable is a distinct output of the action, meaningful in terms of the action?s overall objectives and may be a report, a document, a technical diagram, a software, etc.}\footnote{Deliverable numbers ordered according to delivery dates. Please use the numbering convention <WP number>.<number of deliverable within that WP>. For example, deliverable 4.2 would be the second deliverable from work package 4.}
\item List of major milestones;\footnote{Milestones are control points in the action that help to chart progress. Milestones may correspond to the completion of a key deliverable, allowing the next phase of the work to begin. They may also be needed at intermediary points so that, if problems have arisen, corrective measures can be taken. A milestone may be a critical decision point in the action where, for example, the researcher must decide which of several technologies to adopt for further development.}
\item Secondments if applicable.
\end{itemize}
The schedule should be in terms of number of months elapsed from the start of the project.
}

\subsection{Appropriateness of the management structure and procedures, including quality management and risk management}

{\em
Develop your proposal according to the following lines:
\begin{itemize}
\item Project organisation and management structure, including the financial management strategy, as well as the progress monitoring mechanisms put in place;
\item Risks that might endanger reaching project objectives and the contingency plans to be put in place should risk occur.
\end{itemize}
}

\begin{figure}[htbp]
Gantt chart
Reflecting work package, secondments, training events and dissemination / public engagement activities
\begin{center}

\begin{ganttchart}[
    canvas/.append style={fill=none, draw=black!5, line width=.75pt},
    hgrid style/.style={draw=black!5, line width=.75pt},
    vgrid={*1{draw=black!5, line width=.75pt}},
    title/.style={draw=none, fill=none},
    title label font=\bfseries\footnotesize,
    title label node/.append style={below=7pt},
    include title in canvas=false,
    bar label font=\small\color{black!70},
    bar label node/.append style={left=2cm},
    bar/.append style={draw=none, fill=black!63},
    bar progress label font=\footnotesize\color{black!70},
    group left shift=0,
    group right shift=0,
    group height=.5,
    group peaks tip position=0,
    group label node/.append style={left=.6cm},
    group progress label font=\bfseries\small
  ]{1}{24}
  \gantttitle[
    title label node/.append style={below left=7pt and -3pt}
  ]{Month:\quad1}{1}
  \gantttitlelist{2,...,24}{1} \\
  \ganttgroup{Work Package}{1}{10} \\
  \ganttgroup{Deliverable}{5}{15} \\
  \ganttgroup{Milestone}{5}{5} \\
  \ganttgroup{Secondment}{20}{23} \\
  \ganttgroup{Conference}{16}{16} \\
  \ganttgroup{Workshop}{17}{17} \\
  \ganttgroup{Seminar}{18}{18} \\
  \ganttgroup{Dessemination}{23}{24} \\
  \ganttgroup{Public engagement}{4}{5} \\
  \ganttgroup{Other}{7}{10}
\end{ganttchart}

\end{center}
\end{figure}

\subsection{Appropriateness of the institutional environment (infrastructure)}
\label{sec:institution}

{\em
\begin{itemize}
\item Give a description of the main tasks and commitments of the beneficiary and partners (if applicable).
\item Describe the infrastructure, logistics, facilities offered in as far they are necessary for the good implementation of the action.
\end{itemize}
}

The Researcher will be hosted in the Department of Mathematics and Statistics, Lancaster University.  Prof.\ Leslie will provide the main mentorship and research supervision.  The Statistical Learning group will provide further immediate support to the Researcher.  The Department has extremely strong links with research groups in Operations Research in Lancaster University Management School, through the STOR-i Centre for Doctoral Training, and with Computer Science, through the Data Science Institute.  Therefore multiple researchers in cognate areas will contribute to the project with informal mentorship and research leadership.  In terms of physical resources, the Department will provide high quality office space and standard IT facilities to allow the researcher to carry out the project.  IS BIG COMPUTING NEEDED/AVAILABLE?

\subsection{Competences, experience and complementarity of the participating organisations and institutional commitment}
\label{sec:competences}


{\em
The active contribution of the beneficiary to the research and training activities should be described. For GF also the role of partner organisations in Third Countries for the outgoing phase should appear. Additionally a letter of commitment shall also be provided in Section 7 (included within the PDF file of part B, but outside the page limit) for the partner organisations in Third Countries.
NB: Each participant is described in Section 5. This specific information should not be repeated here.
}

The Department of Mathematics and Statistics at Lancaster University was ranked fifth equal in the United Kingdom in the most recent Research Excellence Framework assessment.  The Department has a thriving research environment, with 50 faculty, 11 post-doctoral fellows, and 72 PhD students.  The Department has numerous government- and industry-funded research projects, many of which relate to industrially-motivated statistics and operations research and are related to the currently-proposed project.  The skill set of the Researcher complements that of the Beneficiary by providing expertise in current algorithmic approaches to bandit algorithms and reinforcement learning.  The host institution in return provides expertise in statistical methodology appropriate to online inference, and game theoretical learning, and a strong track-record of working with industry to ensure the fundamental research is relevant and generates impact.

\newpage
\section{CV of the Experienced Researcher}
\label{sec:cv}

This section should be limited to maximum 5 pages and should include the standard academic and research record. Any research career gaps and/or unconventional paths should be clearly explained so that this can be fairly assessed by the independent evaluators.
The Experienced Researchers must provide a list of achievements reflecting their track, and this may include, if applicable:

\begin{enumerate}
\item Publications in major international peer-reviewed multi-disciplinary scientific journals and/or in the leading international peer-reviewed journals, peer-reviewed conference proceedings and/or monographs of their respective research fields, indicating also the number of citations (excluding self-citations) they have attracted.
\item Granted patent(s).
\item Research monographs, chapters in collective volumes and any translations thereof.
\item Invited presentations to peer-reviewed, internationally established conferences and/or international advanced schools.
\item Research expeditions that the Experienced Researcher has led. 
\item Organisation of International conferences in the field of the applicant (membership in the steering and/or programme committee).
\item Examples of participation in industrial innovation.
\item Prizes and Awards.
\item Funding received so far
\item Supervising, mentoring activities
\end{enumerate}

During the course of my studies several invaluable experiences have greatly contributed to my desire to pursue a research-based career in Computer Science. I have had the opportunity to participate in stimulating research projects, in such areas as Machine Learning or Signal Processing.
% As a result, I have obtained a diverse research background which I seek to put in practice through a challenging and thus interesting post-doctoral position at  Learning Agents Research Group. 
From my early years as an undergraduate student I have tried to keep the balance between theory and application. After three years of intensive Mathematics and Physics studies I entered TELECOM SudParis, a Telecommunication engineering school. There, on the one hand my engineering education made me comfortable with programming (C/C++, Java) and Network issues (LANs, WANs) and on the other and I personally got involved in a research project on PCA algorithms which has lead to a publication at ICASSP 2009. In 2008, I continued with my graduate studies in Applied Mathematics as a master student with focus on Statistical Learning where I developed solid background Machine Learning theory (including a course on Graphical Models by Francis Bach and one on Reinforcement Learning by Rémi Munos). Still I completed my master with an internship at INRIA research lab where I applied statistical learning techniques to help design a realistic automatic ad-server for Orange Inc affiliated websites. This work has launched a collaboration which is still in progress.
 
My current research involves the investigation of machine learning techniques to create algorithms that, in some way, adapts to its users, or more generally learns from its environment. The approach is both theoretical and application oriented. A major objective in our algorithms development is to ensure our algorithms capture the real complexity of a problem and testing in practice their performances in real world problems. During my PhD, I investigated Reinforcement Learning (RL) which is a field where one tries to solve complex systems where an agent has to learn from its environment. More precisely, the focus was on a class of algorithms called ``Classification-based Policy Iteration'' (CBPI) which are algorithms that learn directly the policies as output of a classifier. Thus they avoid, as in the standard RL techniques, to define a policy through an associated value function as this value function is often poorly approximated. Therefore, this class of algorithms is expected to perform better than its value-based counterparts whenever the policies are easier to represent than their value functions. However, CBPI algorithms can require large number of samples from the environment. To improve the CBPI efficiency, I proposed new hybrid approaches using value function approximations in the CBPI framework that leverage the benefits of both approaches (which led to two publications in ICML 2011 \& 2012 while a journal paper has been published in JMLR). Moreover, we applied our techniques in the game of Tetris, a domain where RL techniques had obtained poor results, and learned a controller removing on average 50.000.000 lines (the best in the literature, to the best of our knowledge which is reported in a paper in NIPS 2013).

I also investigated Bandit problems. Bandit problems are core problems to model any problem involving adaptiveness. We designed a sampling strategy to solve several bandit problems in parallel (which led to two publications in NIPS 2011 \& 2012).

During the course of my Ph.D. I worked as an research intern for 6 months at Technicolor Labs in Palo Alto California under the supervision of Branislav Kveton.  Our primary goal was to improve the questionnaire asked to elicit movie preferences of users for a recommendation website. The problem was cast as an adaptive submodular maximization problem. The novelty was that we consider this problem in the case where the preferences of the users are not supposed to be known to build the questionnaire but need to be learned (which led to a publication in NIPS 2013).

As a post-doctorate in the Queensland University of Technology, under the supervision of Peter Bartlett, I am conducting research in online learning.  My first project deals with a combinatorial set of possible choices, is set in a stochastic setting and could model network routing problem (online shortest-path problem). The second one is set in the non-stochastic setting (adversarial) where the goal is to give a simple setting of this bandit game that admits an exact mimimax solution. This therefore is a more theoretical question that draws connection with game theory.

Through the experiences already described I developed my ability to work in a team environment. The international conferences, internships and summer schools I have been attending gave me the opportunity to learn and exchange with researchers from diverse horizons. In addition, teaching computer science (Algorithmic with Python \& Databases) for Master and Licence students keeps enriching my communication skills. I build up my programming skills through my curriculum in a telecommunication engineering school and later through the lectures and practical sections I gave. Moreover most of my projects have involved programming part which have made me comfortable with coding in Python and  C++.

My long term career goal is to become a  researcher.  I wish to gain professional experience at an environment that will allow me to expand my knowledge and capabilities through collaborations with researchers who can mentor and inspire me. I am confident that GRASP will provide me with such an environment and much more. It fits my willingness to  lead research that can find real applications, particularly in artificial intelligence for games or robotic purpose (as I already worked on the Tetris game). I believe that my background in Machine Learning will permit me to take on GRASP challenge on designing powerful lifelong learning algorithms. I also believe that my diverse research background, and my prior exposure to similar research environments make me a unique candidate for the internship program at GRASP. I look forward to conducting research at GRASP  world-class research environment, while nurturing my innovative and practical abilities.
 \begin{center} \textbf{Curriculum Vitae of the Applicant, Dr Victor Gabillon}  \end{center}
 
\noindent\textbf{Education}\\[-.4cm]\noindent\makebox[\linewidth]{\rule{\columnwidth}{0.4pt}}\\[.1cm]
\noindent\begin{tabularx}{\columnwidth}{@{} l X @{}}
\noindent\textbf{PhD in Computer Science} (Accessit Award of the AI French Association, AFIA)& \hfill \textbf{June 2014} \\
Team SequeL, INRIA Lille - Nord Europe, France\\
%Defended on June 12, 2014\\
\textit{Title:} ``Budgeted Classification-based Policy Iteration''\\
\textit{Domains:} Reinforcement learning \& Bandits games\\
\textit{Supervisors:}  Mohammad Ghavamzadeh \&  Philippe Preux\\
\noindent\textit{Examiners:}\begin{tabular}{ll}  &Peter Auer (Leoben University), Olivier Cappé   (Télécom ParisTech), \\
\noindent & Shie Mannor   (Technion)  and Csaba Szepesvári  (Alberta  University)  
\end{tabular}\\[.2cm]
\textbf{M.Sc. Image Processing \& Statistical Learning} with honours &\hfill \textbf{ Sep 2009}\\
 École Normale Supérieure, Cachan, France\\[.2cm]
\textbf{Engineering degree in information technology}  &\hfill \textbf{ Sep. 2009}\\
 TELECOM SudParis, Évry, France
\end{tabularx}\\[.2cm]

\noindent\textbf{Professional Activities}\\[-.4cm]\noindent\makebox[\linewidth]{\rule{\columnwidth}{0.4pt}}\\[.1cm]
\noindent\textbf{Postdoctoral Research Fellow in Statistics} \textit{full time } \hfill \textbf{Nov 2015 -- ongoing} \\
School of Mathematical Sciences, Queensland University of Technology, Brisbane, Australia\\
\noindent\textbf{PhD Researcher} \textit{full time } \hfill \textbf{Oct 2009 -- June 2014}\\
Team SequeL, INRIA Lille - Nord Europe, France\\
\textbf{Research Engineer}  \textit{full time }  \hfill \textbf{ Mar 2013 -- Sep 2013}\\
Technicolor Research Group, Palo Alto, USA.\\
\textbf{External Lecturer} \textit{part time } \hfill \textbf{ Fall 2012}\\
Lille 1 University, France\\
\textbf{External Lecturer} \textit{part time }\hfill \textbf{2010 -- 2011  }\\
Lille 3 University, France\\ 
\textbf{Research Engineer}  \textit{full time }  \hfill \textbf{ June 2008 -- Sep 2008}\\
Chinese Academy of Science, Beijing, China.\\[.2cm] 

\noindent\textbf{Awards \& Grants}\\[-.4cm]\noindent\makebox[\linewidth]{\rule{\columnwidth}{0.4pt}}\\[.1cm]
\noindent\textbf{Postdoctoral Research Fellowship in Statistics} \hfill \textbf{Nov 2015} \\
Two-year fellowship funded by the Queensland University of Technology\\
\noindent\textbf{Second place award for the best French PhD in Artificial Intelligence }  \hfill \textbf{June 2015}\\
Award from AFIA, the French Association for Artificial Intelligence.\\
\textbf{Best applied paper award}  \hfill \textbf{ Jan 2010}\\
Award from the EGC conference, French speaking conference on knowledge mining and management.\\
\textbf{PhD Grant}  \hfill \textbf{ Oct 2009}\\
Three-year grant funded by the French Ministry of Research\\[.2cm]

\noindent\textbf{Research Expeditions}\\[-.4cm]\noindent\makebox[\linewidth]{\rule{\columnwidth}{0.4pt}}\\[.1cm]
\noindent\textbf{3 months at Berkeley Statistic Departement, USA} \hfill \textbf{Mar -- June 2015} \\
Hosted by Peter Bartlett\\
\noindent\textbf{One week at Inria Nancy-Grand Est, France} \hfill \textbf{June 2012} \\
Hosted by Bruno Scherrer of the team Maia\\

\noindent\textbf{Peer Reviewer}\\[-.4cm]\noindent\makebox[\linewidth]{\rule{\columnwidth}{0.4pt}}\\[.1cm]
I have been an official reviewer for the Neural Information Processing Systems (NIPS) international conference in 2014 and 2015 and I have reviewed papers for the Machine Learning Jounal and the Journal of Machine Leaning Research (JMLR).\\


\noindent 
\textbf{Invited Presentations}
\\[-.4cm]\noindent\makebox[\linewidth]{\rule{\columnwidth}{0.4pt}}\\[.1cm]
\textbf{\textit{Talks other than Conference presentations}}\\
\textbf{Talk} Oxford Robotics Research Group Seminar, Oxford, UK, May 2014\\
 ``Classification-Based Policy Iteration perform well in the game of Tetris''.\\
\noindent \textbf{Talk} Gatsby Reinforcement Learning Research Group, London, UK, May 2014\\
 ``Classification-Based Policy Iteration perform well in the game of Tetris''.
  
\noindent \textbf{Talk} Team Maia Seminar, Nancy, France, June 2012\\
 ``Pure Exploration Bandits''.
 
\noindent \textbf{Talk} Co-Adapt Seminars, Marseille, France,  May 2012\\
 ``Pure Exploration Bandits for Brain-Computer Interface?''.\\

\noindent\textbf{Publications}\\[-.4cm]\noindent\makebox[\linewidth]{\rule{\columnwidth}{0.4pt}}\\[.1cm]
\textit{\textbf{Peer-reviewed journal article}}\\
\noindent\begin{tabularx}{\columnwidth}{@{} l X @{}}
 J1. & Bruno Scherrer, Mohammad Ghavamzadeh, Victor Gabillon $\&$ Matthieu Geist, \textbf{\emph{Approximate Modified Policy Iteration}}, to appear in Journal of Machine Learning Research (JMLR).
  \end{tabularx}\\
  
\noindent\textit{\textbf{Peer-reviewed conference article}}\\
\noindent\begin{tabularx}{\columnwidth}{@{} l X @{}}
C9. & Victor Gabillon, Branislav Kveton, Zheng Wen, Brian Eriksson $\&$ S. Muthukrishnan, \textbf{\emph{Large Scale Optimistic Adaptive Submodularity}}.
AAAI $2014$, $28^{th}$ Conference of the Association for the Advancement of  Artificial Intelligence.
Oral presentation at Quebec City, Canada, July $2014$.\\

C8. & Victor Gabillon, Mohammad Ghavamzadeh $\&$ Bruno Scherrer, 
\textbf{\emph{Approximate Dynamic Programming Finally Performs Well in the Game of Tetris}}.
NIPS $2013$, $27^{th}$ Conference on Neural Information Processing Systems.
Poster presentation at South Lake Tahoe, Nevada, December $2013$.\\


C7. & Victor Gabillon, Branislav Kveton, Zheng Wen, Brian Eriksson $\&$ S. Muthukrishnan, \textbf{\emph{Adaptive Submodular Maximization in Bandit Setting}}.
NIPS $2013$, $27^{th}$ Conference on Neural Information Processing Systems.
Poster presentation at South Lake Tahoe, Nevada, December $2013$.\\


C6. & Victor Gabillon, Mohammad Ghavamzadeh $\&$  Alessandro Lazaric, \textbf{\emph{Best Arm Identification: A unified approch to fixed budget and fixed confidence}}.
NIPS $2012$, $26^{th}$ Conference on Neural Information Processing Systems.
Poster presentation at South Lake Tahoe, Nevada, December $2012$.\\


C5. & Bruno Scherrer, Mohammad Ghavamzadeh, Victor Gabillon $\&$ Matthieu Geist, \textbf{\emph{Approximate Modified Policy Iteration}}.
ICML $2012$, $29^{th}$  International Conference on Machine Learning.
Long lecture presentation at Edinburgh, Scotland, June $2012$.\\


C4. & Victor Gabillon, Mohammad Ghavamzadeh, Alessandro Lazaric $\&$ Sébastien Bubeck, \textbf{\emph{Multi-Bandit Best Arm Identification}}.
NIPS $2011$, $25^{th}$ Conference on Neural Information Processing Systems.
Poster presentation at Granada, Spain, December $2011$.\\


C3. & Victor Gabillon, Alessandro Lazaric, Mohammad Ghavamzadeh $\&$  Bruno Scherrer, \textbf{ \emph{Classification-based Policy Iteration with a Critic}}. ICML $2011$, $28^{th}$  International Conference on Machine Learning. Lecture presentation at Bellevue, USA, June $2011$.\\

C2. & Victor Gabillon, Jérémie Mary $\&$ Philippe Preux, \textbf{ \emph{Affichage de publicités sur des portails web}}. EGC $2010$, $10^{th}$ French-speaking International Conference on Knowledge Extraction and Management. Lecture presentation of long article at Hammamet, Tunisia, January $2010$. Best applied paper award.\\

 
C1. & Jean-Pierre Delmas $\&$ Victor Gabillon,\textbf{ \emph{Asymptotic performance analysis of PCA algorithms based on the weighted subspace criterion}}.  ICASSP $2009$, International Conference on Acoustics, Speech and Signal Processing. Poster presentation at Taipei, Taiwan, April $2009$. 
   \end{tabularx}\\

\noindent\textit{\textbf{Peer-reviewed workshop article}}\\
\noindent\begin{tabularx}{\columnwidth}{@{} l X @{}}
 W1. & Victor Gabillon,  Alessandro Lazaric $\&$ Mohammad Ghavamzadeh, \textbf{ \emph{Rollout Allocation Strategies for Classification-based Policy Iteration}}. Workshop on Reinforcement Learning and Search in Very Large Spaces International Conference on Machine Learning,  Lecture presentation at Haifa, Israel, June $2010$.
  \end{tabularx}\\
  
  

\noindent\textit{\textbf{Major research achievements \& industrial innovations}}\\[-.4cm]\noindent\makebox[\linewidth]{\rule{\columnwidth}{0.4pt}}\\[.1cm]
\begin{itemize}
\item \textit{\text{Research:}} \textit{\textbf{Reinforcement Learning is finally competitive:}} We proposed a new family of reinforcement learning methods based on the ``Classification-based Policy Iteration'' algorithms. In addition to proposing theoretical analysis of this methods (C3,C5), we implemented a complex and extensive experimental studies of the performance of this algorithms in the famous benchmark of the game of Tetris. Our result show that for the first time a reinforcement learning methods performs well in Tetris even improving on the state-of-the-art techniques. Moreover, while these state-of-the-art techniques were based on black-box optimisation techniques that requires a lots of samples from the environment, our methods require 10 times less samples to learn Tetris strategies with same performance (C8).
\item \textit{\text{Industry:}} \textit{\textbf{Constrained Learning for Orange Ad Server:}} Orange, the french leading company in telecommunications had made a contract with the research team SequeL in order to turn their online web-advertising services automatic. My initial goal was to make a survey of the machine learning literature and find an appropriate solution optimizing their clic-per-rate revenues. This solution had to take into account specific new constraints on the limited and known number of display per ads. Finally, a new approach was proposed combining linear programming and bandits algorithms with experiments on synthetic data. The results was published and awarded in a french speaking conference (C2) and started a collaboration between SequeL and Orange which is still running.  
\item \textit{\text{Industry:}} \textit{\textbf{Adaptive Questionnaire Design at Technicolor Inc:}} During research and developpement internship at Technicolor, the primary goal was to improve the questionnaire asked to elicit movie preferences of users for a recommendation website. The problem was cast as an adaptive submodular maximization problem. The novelty was that we considered this problem in the case where the preferences of the users are unknown  but need to be learned in order to build an adaptive questionnaire (C7,C9).
\end{itemize}



\noindent\text{\textbf{Teaching}}\\[-.4cm]\noindent\makebox[\linewidth]{\rule{\columnwidth}{0.4pt}}\\[.1cm]
In the past 5 years I taught 216 hours of undergraduate and master's courses in France.\\
\textit{\textbf{Instructor:}} 
 \begin{itemize} 
 \item\textit{Introduction to algorithmic and programming with Python.}\\
    $48$ hours (lectures and practical sessions). Winter 2010, Fall 2011 \& Fall 2012\\
    $1^{rst}$ year of Master \textit{Computer science and document} at Lille $3$ University and  $1^{rst}$ year of Licence \textit{Physics-Chemistry} at Lille $1$ University.
    \end{itemize}
\textit{\textbf{Teaching assistant:}} 
\begin{itemize}
 \item \textit{SQL and Python.} $36$ hours (practical sessions). Fall 2010.\\
    $3^{rd}$ year of Licence \textit{Mathematics and computer science applied to social sciences} at Lille 3 University. 
\item \textit{Designing databases and object-oriented programming}.
    $36$ hours (practical sessions). Winter 2011.\\
    $3^{rd}$ year of Licence \textit{Mathematics and computer science applied to social sciences} at Lille 3 University. 
\end{itemize}


\newpage
\section{Capacities of the Participating Organisations}
\label{sec:capacities}

%All organisations (whether beneficiary or partner organisation) must complete the appropriate table below, which will give input on the profile of the organisation as a whole. Complete one table of maximum one page for the beneficiary and half a page per partner organisation (min font size: 9). The experts will be instructed to disregard content above this limit.
\vspace{\baselineskip}

{\fontsize{9bp}{1em}\selectfont % should be 9pt
\noindent\begin{tabular}{>{\raggedright}p{.25\textwidth}p{.7\textwidth}}
  \multicolumn{2}{l}{\textbf{Beneficiary: Lancaster University}} \\\midrule
\textbf{General Description} & Lancaster University is a top ten UK university.  The Department of Mathematics and Statistics, within the Faculty
of Science and Technology, hosts one of the largest and strongest statistics research groups in the
UK comprising 25 academic staff, 10 research associates and around 50 FTE research students. In the 2014
Research Excellence Framework assessment, the Mathematical Sciences at Lancaster were ranked fifth overall and third in
terms of the impact of research.  Research is supported by grants from the UK Research Councils, the European Commission, and industrial sponsors. The statistics research group is also a fundamental partner in Lancaster's new Data Science Institute, which aims to act as a catalyst for Data Science, providing an end-to-end interdisciplinary research capability --- from infrastructure and fundamentals through to globally relevant problem domains and the social, legal and ethical issues raised by the use of Data Science. 

\\\midrule
\textbf{Role and Commitment of key persons (supervisor)} &
Prof.\ David Leslie, PhD in Mathematics (University of Bristol, 2003).  17 PhD students and 2 post-doctoral fellows supervised. 5\% FTE time commitment to the project throughout the 24 month duration.
\\\midrule
\textbf{Key Research Facilities, Infrastructure and Equipment} &
The Department of Mathematics and Statistics is housed in dedicated space at Lancaster University.  The researcher will be provided with office space and basic equipment within the Department. \TODO{Computing facilities?}


\\\midrule
\textbf{Independent research premises?} & Yes

\\\midrule
\textbf{Previous Involvement in Research and Training Programmes} &
Between 2001 and 2005 the department held the Marie Curie Training Site status for its PhD programme. The Postgraduate Statistics Center (PSC) was founded in 2005 as the only Centre for Excellence in Teaching and Learning focussing on postgraduate statistics in the UK. The PSC is still operative and runs three Masters degrees (Statistics, Quantitative Methods, and Quantitative Finance) and coordinates the PhD programme in statistics.


\\\midrule
\textbf{Current involvement in Research and Training Programmes} &
Together with the Management School, the Department hosts and runs STOR-i, a
multi-million pound EPSRC-funded Centre for Doctoral Training in
Statistics and Operational Research in partnership with industry.  The
Centre was established in 2010 and funds 12 PhD students per year.  The department is also a key player in the Academy for Phd Training in Statistics, a collaboration between major UK statistics research groups to organise courses for first-year PhD students in statistics and applied probability nationally.  The group hosts one node of a multi-institution Programme Grant on Intractable Likelihood, and received industrial funding from companies including Shell, BT, Google and Unilever \TODO{Other big grants?}. The
Department's Medical and Pharmaceutical Statistics Research Unit
works closely with the pharmaceutical
industry and public sector research institutes to develop novel
statistical methods for the design and analysis of clinical trials. It
leads the EU-funded research training network IDEAS (www.ideas-itn.eu) and
is an integral part of the Medical Research Council funded North-West Hub for Trials
Methodology Research.
\\\midrule
\textbf{Relevant Publications and/or research/innovation products} &

 Perkins, S. and Leslie, D.S. (2014)  Stochastic fictitious play with continuous action sets. {\em Journal of Economic Theory} {\bf 152}, 179--213.

Chapman, A.C., Leslie, D.S., Rogers, A. and Jennings, N.R. (2013) Convergent learning algorithms for unknown reward games. {\em SIAM Journal on Control and Optimization} {\bf 51}, 3154-3180.

% Chapman, A.C., Leslie, D.S., Rogers, A. and Jennings, N.R. (2013) Learning in unknown reward games: Application to sensor networks.  {\em The Computer Journal} bxt082.

% Perkins, S. and Leslie, D.S. (2012) Asynchronous stochastic approximation with differential inclusions.  {\em Stochastic Systems} {\bf 2}, 409--446.
May, B.C., Korda, N., Lee, A. and Leslie, D.S. (2012) Optimistic Bayesian sampling in contextual-bandit problems. {\em Journal of Machine Learning Research} {\bf 13}, 2069--2106.

% Chapman, A.C., Rogers, A.C., Jennings, N.R. and Leslie, D.S. (2011)  A unifying framework for iterative approximate best response algorithms for distributed constraint optimisation problems.  {\em The Knowledge Engineering Review} {\bf 26}, 411--444.

Larsen, T., Leslie, D.S., Collins, E.J. and Bogacz, R. (2010) Posterior weighted reinforcement learning with state uncertainty.  {\em Neural Computation} {\bf 22}, 1149--1179.

% Rezek, I., Leslie, D.S., Reece, S., Roberts, S.J., Rogers, A., Dash, R.K., and Jennings, N.R. (2008) On similarities between inference in game theory and machine learning. {\em Journal of Artificial Intelligence Research} {\bf 33}, 259--283.

% {Leslie, D.S.} and Collins, E.J. (2006)  Generalised weakened fictitious    play.  {\em Games and Economic Behavior} {\bf 56}, 285--298.
% {Leslie, D.S.} and Collins, E.J. (2005)  Individual Q-learning in normal form games. {\em SIAM Journal on Control and Optimization} {\bf 44},    495--514.

{Leslie, D.S.} and Collins, E.J. (2003) Convergent multiple-timescales    reinforcement learning algorithms in normal form games. {\em Annals of    Applied Probability} {\bf 13}, 1231--1251.

\\\bottomrule
\end{tabular}}
\vspace{\baselineskip}

%{\fontsize{9bp}{1em}\selectfont
%\noindent\begin{tabular}{>{\raggedright}p{.25\textwidth}p{.7\textwidth}}
%  \multicolumn{2}{l}{\textbf{Partner Organisation Y}} \\\midrule
%\textbf{General Description} &
%
%\\\midrule
%\textbf{Key Persons and Expertise (supervisor)} &
%
%\\\midrule
%\textbf{Key Research facilities, infrastructure and equipment} &
%
%\\\midrule
%\textbf{Previous and Current Involvement in Research and Training Programmes} &
%
%\\\midrule
%\textbf{Relevant Publications and/or research/innovation product} &
%(Max 3)
%\\\bottomrule
%\end{tabular}}


\newpage
\vspace{15mm}
\begin{center}


        \Large{
      
     
        \textbf{ENDPAGE}
  
          \vspace{15mm}
          MARIE SKLODOWSKA-CURIE ACTIONS\\
          \vspace{1cm}
          
          \textbf{Individual Fellowships (IF)}\\
          \textbf{Call: H2020-MSCA-IF-2014}
          \vspace{2cm}                   

          PART B
          \vspace{2.5cm}

          ``\acronym''
          \vspace{2cm}

          \textbf{This proposal is to be evaluated as:}
          \vspace{.5cm}

          \textbf{[Standard EF]}
        }

  \end{center}
\vspace{1cm}


\end{document}